\documentclass[12pt]{book}
\parindent=0px
\usepackage{amssymb, amsmath, fullpage, graphicx}
\usepackage{enumitem, siunitx}
\usepackage[utf8]{inputenc}
\usepackage[T1]{fontenc}

\newcommand{\set}[1]{\{#1\}}
\newcommand{\Set}{\text}
\newcommand{\defn}[1]{\textbf{#1}}
\newcommand{\qed}{\blacksquare}
\newcommand{\comp}{\overline}
\newcommand{\solution}[2]{\item #1\\ \textbf{Solution}: #2}
\newcommand{\thm}[1]{\underline{\textsc{#1}}}
\newcommand{\env}[2]{\begin{#1} #2 \end{#1}}

\begin{document}
\chapter{Sets and Relations}

\section{Cantor's Concept of a Set}
A \defn{set} $S$ is any collection of definite, distinguishable objects of our intuition or of our intellect to be conceived as a whole. The objects are called the \defn{elements} or \defn{members} of $S$.

\section{The Basis of Intuitive Set Theory}
\defn{Membership relation}: $x \in A$ if the object $x$ is a member of the set $A$. If $x$ is not a member of $A$ then $x \notin A$. $x_1, x_2, \dots, x_n \in A$ is shorthand for $x_1 \in A \wedge x_2 \in A \wedge \dots \wedge x_n \in A$.\\
\defn{The intuitive principle of extension}: Two sets are equal iff they have the same members.
\defn{Set equality}: The equality of two sets $X$ and $Y$ will be denoted by $X = Y$ and inequality of $X$ and $Y$ by $X \neq Y$. Among the basic properties of this relation are:
\begin{gather*}
X = X,\\X = Y \Rightarrow Y = X,\\ X = Y \wedge Y = Z \Rightarrow X = Z,
\end{gather*}
for all sets $X, Y$, and $Z$.\\
\defn{unit set}: a set $\set{x}$ whose sole member is $x$.\\
\defn{collection of sets}: a set whose members are sets.\\
\defn{The intuitive principle of abstraction}: A formula $P(x)$ defines a set $A$ by the convention that the members of $A$ are exactly those objects $a$ such that $P(a)$ is a true statement, denoted by $A = \set{x \mid P(x)}$.\\
Note: $\set{x \in A \mid P(x)} := \set{x \mid x \in A \wedge P(x)}$. For a property $P$ and function $f$ we can write $\set{f(x)\mid P(x)} := \set{y \mid \exists x\colon P(x) \wedge y = f(x)}$.

\section{Inclusion}
If $A$ and $B$ are sets, then $A$ is \defn{included in} $B$ iff each member of $A$ is a member of $B$. Symbolized: $A \subseteq B$. We also say that $A$ is a \defn{subset} of $B$. Equivalently, $B$ \defn{includes} $A$, symbolized by $B \supseteq A$.\\ The set $A$ is \defn{properly included in} $B$ ($A$ is a \defn{proper subset} of $B$ / $B$ \defn{properly includes} $A$) iff $A \subseteq B$ and $A \neq B$.\\ Among the basic properties of the inclusion relation are
\begin{gather*}
X \subseteq X;\\ X \subseteq Y \wedge Y \subseteq Z \Rightarrow X \subseteq Z;\\ X \subseteq Y \wedge Y \subseteq X \Rightarrow X = Y.
\end{gather*}
\defn{empty set}: $\set{x \in A \mid x \neq x}$ for any set $A$ is the set with no elements, symbolized by $\emptyset$.\\
\defn{power set}: the set of all subsets of a given set. $\mathcal{P}(A) = \set{B \mid B \subseteq A}$ for a given set $A$.

\section{Operations for Sets}
\defn{union}: for sets $A$ and $B$, the set of all objects which are members of either $A$ or $B$. $A \cup B = \set{x \mid x \in A \text{ or } x \in B}$. (\defn{sum}/\defn{join})\\
\defn{intersection}: for sets $A$ and $B$, the set of all objects which are members of both $A$ and $B$. $A \cap B = \set{x \mid x \in A \text{ and } x \in B}$. (\defn{product}/\defn{meet})\\

\thm{Lemma}: For every pair of sets $A$ and $B$ the following inclusions hold: $$\emptyset \subseteq A \cap B \subseteq A \subseteq A \cup B.$$
\thm{Proof}: Take $x \in \emptyset$. Since this is false, we can conclude $x \in A \cap B$ and so $\emptyset \subseteq A \cap B$. Now take $x \in A \cap B$. Then $x \in \set{y \mid y \in A \text{ and } y \in B}$ and so $x \in \set{y \mid y \in A} = A$ and thus $A \cap B \subseteq A$. Now take $x \in A$. Then we must have $x \in \set{y \mid y \in A \text{ or } y \in B} = A \cup B$. Then $A \subseteq A \cup B$. $\qed$\\

\defn{disjoint}: $A \cap B = \emptyset$ for sets $A$ and $B$.\\
\defn{intersect}: $A \cap B \neq \emptyset$ for sets $A$ and $B$.\\
\defn{disjoint collection}: for a collection of sets, each distinct pair of its member sets is disjoint.\\
\defn{partition}: for a set $X$, a disjoint collection $\mathcal{A}$ of nonempty and distinct subsets of $X$ such that each member of $X$ is a member of some (exactly one) member of $\mathcal{A}$.\\
\defn{absolute complement} of $A$: $\comp{A} = \set{x \mid x \notin A}$, the set of all members which are not in $A$.\\
\defn{relative complement} of $A$ with respect to $X$: $X - A = X \cap \comp{A} = \set{x \in X \mid x \notin A}$, the set of those members of $X$ which are not members of $A$.\\
\defn{symmetric difference} of $A$ and $B$: $A + B = (A - B) \cup (B - A)$.\\
\defn{universal set}: the set $U$ such that all sets under consideration in a certain discussion are subsets of $U$.\\

\newpage

\section{The Algebra of Sets}
\defn{identities}: equations which are true whatever the universal set $U$ and no matter what particular subsets the letters (other than $U$ and $\emptyset$) represent.\\ \linebreak
\thm{Theorem 5.1}: For any subsets $A, B, C$ of a set $U$ the following equations are identities. Here $\comp{A}$ is an abbreviation for $U - A$.\\
\begin{tabular}{ll}
1. $A \cup (B \cup C) = (A \cup B) \cup C$.& 1'. $A \cap (B \cap C) = (A \cap B) \cap C$.\\
2. $A \cup B = B \cup A$.& 2'. $A \cap B = B \cap A$.\\
3. $A \cup (B \cap C) = (A \cup B) \cap (B \cup C)$.& 3'. $A \cap (B \cup C) = (A \cap B) \cup (A \cap C)$.\\
4. $A \cup \emptyset = A$.& 4'. $A \cap U = A$.\\
5. $A \cup \comp{A} = U$& 5'. $A \cap \comp{A} = \emptyset$.
\end{tabular}\\ \linebreak
\thm{Proof}:\\
\small \underline{Lemma}: Let $X,Y$ be subsets of $U$. Then $X \subseteq X \cup Y$ and $X \subseteq Y \cup X$.\\ \underline{Proof}: Assume $x \in X$ and $x \notin X \cup Y$. Then $x \notin \set{z \mid z \in X \text{ or } z \in Y} \Rightarrow x \notin \set{z \mid z \in X} = X$ which is a contradiction. Now assume $x \in X$ and $x \notin Y \cup X$. Then $x \notin \set{z \mid z \in Y \text{ or } z \in X} \Rightarrow x \notin \set{z \mid z \in X} = X$ which is a contradiction. \normalsize
\renewcommand{\labelenumi}{\arabic{enumi}.}
\begin{enumerate}
\item Assume $x \in A \cup (B \cup C)$. Then $x \in A$ or $x \in B \cup C$. If $x \in A$ then $x \in A \cup B$ and so $x \in (A \cup B) \cup C$. Otherwise if $x \in B \cup C$ then $x \in B$ or $x \in C$. If $x \in B$ then $x \in (A \cup B)$ and so $x \in (A \cup B) \cup C$. If $x \in C$ then $x \in (A \cup B) \cup C$. Therefore $A \cup (B \cup C) \subseteq (A \cup B) \cup C$.\\ Now assume $x \in (A \cup B) \cup C$. Then $x \in A \cup B$ or $x \in C$. If $x \in A \cup B$ then $x \in A$ or $x \in B$. If $x \in A$ then $x \in A \cup (B \cup C)$. If $x \in B$ then $x \in (B \cup C)$ and so $x \in A \cup (B \cup C)$. Otherwise if $x \in C$ then $x \in B \cup C$ and so $x \in A \cup (B \cup C)$. Therefore $(A \cup B) \cup C \subseteq A \cup (B \cup C)$. Hence $A \cup (B \cup C) = (A \cup B) \cup C$.
\renewcommand{\labelenumi}{\arabic{enumi}'.}
\setcounter{enumi}{0}
\item Assume $x \in A \cap (B \cap C)$. Then $x \in A$ and $x \in B \cap C$. Since $x \in B \cap C$ we have $x \in B$ and $x \in C$. Then since $x \in A$ and $x \in B$ we have $x \in A \cap B$. Since $x \in C$ we have $x \in (A \cap B) \cap C$. Therefore $A \cap (B \cap C) \subseteq (A \cap B) \cap C$.\\ Now assume $x \in (A \cap B) \cap C$. Then $x \in A \cap B$ and $x \in C$. Since $x \in A \cap B$ we have $x \in A$ and $x \in B$. Then since $x \in B$ and $x \in C$ we have $x \in B \cap C$. Since $x \in A$ we have $x \in A \cap (B \cap C)$. Therefore $(A \cap B) \cap C \subseteq A \cap (B \cap C)$. Hence $A \cap (B \cap C) = (A \cap B) \cap C$.
\renewcommand{\labelenumi}{\arabic{enumi}.}
\setcounter{enumi}{1}
\item Assume $x \in A \cup B$. Then $x \in A$ or $x \in B$. In either case $x \in B \cup A$ and so $A \cup B \subseteq B \cup A$. Now assume $x \in B \cup A$. Then $x \in B$ or $x \in A$. In either case $x \in A \cup B$ and so $B \cup A \subseteq A \cup B$. Hence $A \cup B = B \cup A$.
\renewcommand{\labelenumi}{\arabic{enumi}'.}
\setcounter{enumi}{1}
\item Assume $x \in A \cap B$. Then $x \in A$ and $x \in B$ and so $x \in B \cap A$. Therefore $A \cap B \subseteq B \cap A$. Now assume $x \in B \cap A$. Then $x \in B$ and $x \in A$ and so $x \in A \cap B$. Therefore $B \cap A \subseteq B \cap A$. Hence $A \cap B = B \cap A$.
\renewcommand{\labelenumi}{\arabic{enumi}.}
\setcounter{enumi}{2}
\item Assume $x \in A \cup (B \cap C)$. Then $x \in A$ or $x \in B \cap C$. If $x \in A$ then $x \in A \cup B$ and $x \in A \cup C$ and so $x \in (A \cup B) \cap (A \cup C)$. Otherwise if $x \in B \cap C$ then $x \in B$ and $x \in C$. Since $x \in B$ we have $x \in A \cup B$. Since $x \in C$ we have $x \in A \cup C$. Then $x \in (A \cup B) \cap (A \cup C)$ and therefore $A \cup (B \cap C) \subseteq (A \cup B) \cap (A \cup C)$.\\ Now assume $x \in (A \cup B) \cap (A \cup C)$. Then $x \in A \cup B$ and $x \in A \cup C$. Since $x \in A \cup B$ we have $x \in A$ or $x \in B$. If $x \in A$ then $x \in A \cup (B \cap C)$. Otherwise if $x \in B$ then $x \in A \cup B$. Since $x \in A \cup C$ we also have that $x \in A$ or $x \in C$. If $x \in A$ then $x \in A \cup (B \cap C)$. Otherwise if $x \in C$ then since $x \in B$ we have $x \in B \cap C$ and so $x \in A \cup (B \cap C)$. Therefore $(A \cup B) \cap (A \cup C) \subseteq A \cup (B \cap C)$. Hence $A \cup (B \cap C) = (A \cup B) \cap (A \cup C)$.
\renewcommand{\labelenumi}{\arabic{enumi}'.}
\setcounter{enumi}{2}
\item Assume $x \in A \cap (B \cup C)$. Then $x \in A$ and $x \in B \cup C$. If $x \in B$ then since $x \in A$ we have $x \in A \cap B$ and so $x \in (A \cap B) \cup (A \cap C)$. Otherwise if $x \in C$ then since $x \in A$ we have $x \in A \cap C$ and so $x \in (A \cap B) \cup (A \cap C)$. Therefore $A \cap (B \cup C) \subseteq (A \cap B) \cup (A \cap C)$. Now assume $x \in (A \cap B) \cup (A \cap C)$. Then $x \in A \cap B$ or $x \in A \cap C$. If $x \in A \cap B$ then $x \in A$ and $x \in B$. Since $x \in B$ we have $x \in B \cup C$. Since we also have $x \in A$ then $x \in A \cap (B \cup C)$. Otherwise if $x \in A \cap C$ then $x \in A$ and $x \in C$. Since $x \in C$ we have $x \in B \cup C$. Since we also have $x \in A$ then $x \in A \cap (B \cup C)$. Therefore $(A \cap B) \cup (A \cap C) \subseteq A \cap (B \cup C)$. Hence $A \cap (B \cup C) = (A \cap B) \cup (A \cap C)$.
\renewcommand{\labelenumi}{\arabic{enumi}.}
\setcounter{enumi}{3}
\item Assume $x \in A \cup \emptyset$. Then $x \in A$ or $x \in \emptyset$. Since $x \in \emptyset$ is impossible, we must have the $x \in A$ and so $A \cup \emptyset \subseteq A$. Now assume $x \in A$ then $x \in A \cup \emptyset$ and so $A \subseteq A \cup \emptyset$. Hence $A \cup \emptyset = A$.
\renewcommand{\labelenumi}{\arabic{enumi}'.}
\setcounter{enumi}{3}
\item Assume $x \in A \cap U$. Then $x \in A$ and $x \in U$. Therefore $A \cap U \subseteq A$. Now assume $x \in A$. Then since $A \subseteq U$ we have $x \in U$ and so $x \in A \cap U$. Therefore $A \subseteq A \cap U$. Hence $A \cap U = A$.
\renewcommand{\labelenumi}{\arabic{enumi}.}
\setcounter{enumi}{4}
\item Assume $x \in A \cup \comp{A}$. Then $x \in A$ or $x \in \comp{A}$. Since $A \subseteq U$ and $\comp{A} \subseteq U$ in either case we have $x \in U$ and so $A \cup \comp{A} \subseteq U$. Now assume $x \in U$. Then $x \in A$ or $X \notin A$ for any set $A$. Thus $x \in A$ or $x \in \comp{A}$ and so $x \in A \cup \comp{A}$. Therefore $U \subseteq A \cup \comp{A}$. Hence $A \cup \comp{A} = U$.
\renewcommand{\labelenumi}{\arabic{enumi}'.}
\setcounter{enumi}{4}
\item Assume $x \in A \cap \comp{A}$. Then $x \in A$ and $x \in \comp{A}$. Since $x \in \comp{A}$ we have $x \notin A$. Since $x \in A$ and $x \notin A$ we have $x \in \emptyset$. Therefore $A \cap \comp{A} \subseteq \emptyset$. Since $\emptyset \subseteq X$ for any set $X$ we have $\emptyset \subseteq A \cap \comp{A}$. Hence $A \cap \comp{A} = \emptyset$. $\qed$
\end{enumerate}

\underline{General associative law for set union}: The sets obtainable from given sets $A_1, A_2, \dots, A_n$ in that order, by use of the operation of union are all equal to one another. The set defined by $A_1, A_2, \dots, A_n$ in this way will be written as $$A_1 \cup A_2 \cup \dots \cup A_n.$$

\underline{General associative law for set intersection}: The sets obtainable from given sets $A_1, A_2, \dots, A_n$ in that order, by use of the operation of intersection are all equal to one another. The set defined by $A_1, A_2, \dots, A_n$ in this way will be written as $$A_1 \cap A_2 \cap \dots \cap A_n.$$

\underline{General commutative law for set union}: If $1',2',\dots,n'$ are $1,2,\dots,n$ in any order, then $$A_1 \cup A_2 \cup \dots \cup A_n = A_{1'} \cup A_{2'} \cup \dots \cup A_{n'}.$$

\underline{General commutative law for set intersection}: If $1',2',\dots,n'$ are $1,2,\dots,n$ in any order, then $$A_1 \cap A_2 \cap \dots \cap A_n = A_{1'} \cap A_{2'} \cap \dots \cap A_{n'}.$$

\underline{General distributive law for set union}: $$A \cup (B_1 \cap B_2 \cap \dots \cap B_n) = (A \cup B_1) \cap (A \cup B_2) \cap \dots \cap (A \cup B_n).$$

\underline{General distributive law for set intersection}: $$A \cap (B_1 \cup B_2 \cup \dots \cup B_n) = (A \cap B_1) \cup (A \cap B_2) \cup \dots \cup (A \cap B_n).$$

\defn{dual}: An equation, or an expression, or a statement within the framework of the algebra of sets obtained from another by interchanging $\cup$ and $\cap$ along with $\emptyset$ and $U$.\\

\defn{principle of duality} for the algebra of sets: If $T$ is any theorem expressed in terms of $\cup$, $\cap$, and $\comp{\phantom{A}}$, then the dual of $T$ is also a theorem.\\

\thm{Theorem 5.2}: For all subsets $A$ and $B$ of a set $U$, the following statements are valid. Here $\comp{A}$ is an abbreviation for $U - A$.\\
\begin{tabular}{ll}
6. If, for all $A$, $A \cup B = A$, then $B = \emptyset$.& 6'. If, for all $A$, $A \cap B = A$ then $B = U$.\\
7,7'. If $A \cup B = U$ and $A \cap B = \emptyset$, then $B = \comp{A}$.\\
8,8'. $\comp{\comp{A}} = A$.\\
9. $\comp{\emptyset} = U$.& 9'. $\comp{U} = \emptyset$.\\
10. $A \cup A = A$& 10'. $A \cap A = A$.\\
11. $A \cup U = U$& 11'. $A \cap \emptyset = \emptyset$.\\
12. $A \cup (A \cap B) = A$. &12'. $A \cap (A \cup B) = A$.\\
13. $\comp{A \cup B} = \comp{A} \cap \comp{B}$& 13'. $\comp{A \cap B} = \comp{A} \cup \comp{B}$.
\end{tabular}\\

\thm{Proof}:
\begin{enumerate}
\renewcommand{\labelenumi}{\arabic{enumi}.}
\setcounter{enumi}{5}
\item Assume $A \cup B = A$ for all $A$. Take $A = \emptyset$. Then $\emptyset \cup B = \emptyset$. Then if $x \in B$ we have $x \in \emptyset \cup B = \emptyset$ and so $B \subseteq \emptyset$. Since $\emptyset \subseteq B$ we have $B = \emptyset$.
\renewcommand{\labelenumi}{\arabic{enumi}'.}
\setcounter{enumi}{5}
\item Assume $A \cap B = A$ for all $A$. Take $A = U$. Then $U \cap B = U$. Then if $x \in U$ we have $x \in U \cap B$ and so $x \in B$. Therefore $U \subseteq B$. Since $B \subseteq U$ we have $B = U$.
\renewcommand{\labelenumi}{\arabic{enumi},\arabic{enumi}'.}
\setcounter{enumi}{6}
\item Assume $A \cup  B = U$ and $A \cap B = \emptyset$ for sets $A$ and $B$. Take $x \in B$. Assume $x \in A$. Then $x \in A \cap B = \emptyset$. By contradiction we have $x \in \comp{A}$. Then $B \subseteq \comp{A}$. Now take $x \in \comp{A}$. Then $x \notin A$. Assume $x \notin B$. Then $x \notin A \cup B = U$. By contradiction we have $x \in B$. Therefore $\comp{A} \subseteq B$ and so $B = \comp{A}$.
\item Take $x \in \comp{\comp{A}}$ for a set $A$. Then $x \notin \comp{A}$ and so $x \in A$. Therefore $\comp{\comp{A}} \subseteq A$. Now take $x \in A$. Then $x \notin \comp{A}$ and so $x \in \comp{\comp{A}}$. Therefore $A \subseteq \comp{\comp{A}}$ and so $\comp{\comp{A}} = A$.
\renewcommand{\labelenumi}{\arabic{enumi}.}
\setcounter{enumi}{8}
\item Take $x \in \comp{\emptyset}$. Then $x \in U \cap \comp{\emptyset}$ and so $x \in U$. Then $\comp{\emptyset} \subseteq U$. Now take $x \in U$. Then $x \notin \emptyset$ and so $x \in U \cap \comp{\emptyset} = U - \emptyset = \comp{\emptyset}$. Therefore $U \subseteq \comp{\emptyset}$ and so $\comp{\emptyset} = U$.
\renewcommand{\labelenumi}{\arabic{enumi}'.}
\setcounter{enumi}{8}
\item Take $x \in \comp{U}$. Then $x \in U \cap \comp{U}$ so $x \in U$ and $x \notin U$. By contradiction $x \in \emptyset$ and so $\comp{U} \subseteq \emptyset$. Now take $x \in \emptyset$. Then $x \notin U$ and so $x \in \comp{U}$. Therefore $\emptyset \subseteq \comp{U}$ and so $\comp{U} = \emptyset$.
\renewcommand{\labelenumi}{\arabic{enumi}.}
\setcounter{enumi}{9}
\item Take $x \in A \cup A$. Then $x \in A$ or $x \in A$ and so $x \in A$. Thus $A \cup A \subseteq A$. Now take $x \in A$. Then $x \in A \cup A$ and so $A \subseteq A \cup A$. Therefore $A \cup A = A$.
\renewcommand{\labelenumi}{\arabic{enumi}'.}
\setcounter{enumi}{9}
\item Take $x \in A \cap A$. Then $x \in A$ and so $A \cap A \subseteq A$. Now take $x \in A$. Then $x \in A \cap A$ and so $A \subseteq A \cap A$. Therefore $A \cap A = A$.
\renewcommand{\labelenumi}{\arabic{enumi}.}
\setcounter{enumi}{10}
\item Take $x \in A \cup U$. Then $x \in A$ or $x \in U$. If $x \in A$ then $x \in U$ since $A \subseteq U$. Therefore $A \cup U \subseteq U$. Now take $x \in U$. Then $x \in A \cup U$ and so $U \subseteq A \cup U$. Therefore $A \cup U = U$.
\renewcommand{\labelenumi}{\arabic{enumi}'.}
\setcounter{enumi}{10}
\item Take $x \in A \cap \emptyset$. Then $x \in \emptyset$ and so $A \cap \emptyset \subseteq \emptyset$. Now take $x \in \emptyset$. Then $x \in A$ (ex falso quodlibet). Thus $x \in A \cap \emptyset$ and so $\emptyset \subseteq A \cap \emptyset$. Therefore $A \cap \emptyset = \emptyset$.
\renewcommand{\labelenumi}{\arabic{enumi}.}
\setcounter{enumi}{11}
\item Take $x \in A \cup (A \cap B)$. Then $x \in A$ or $x \in A \cap B$. If $x \in A \cap B$ then $x \in A$. Therefore $A \cup (A \cap B) \subseteq A$. Now take $x \in A$. Then $x \in A \cup (A \cap B)$. Thus $A \subseteq A \cup (A \cap B)$ and so $A \cup (A \cap B) = A$.
\renewcommand{\labelenumi}{\arabic{enumi}'.}
\setcounter{enumi}{11}
\item Take $x \in A \cap (A \cup B)$. Then $x \in A$ and so $A \cap (A \cup B) \subseteq A$. Now take $x \in A$. Then $x \in A \cup B$ and so $x \in A \cap (A \cup B)$. Therefore $A \subseteq A \cap (A \cup B)$ and so $A \cap (A \cup B) = A$.
\renewcommand{\labelenumi}{\arabic{enumi}.}
\setcounter{enumi}{12}
\item Take $x \in \comp{A \cup B}$. Then $x \notin A \cup B$ and so $x \notin A$ and $x \notin B$. Then $x \in \comp{A}$ and $x \in \comp{B}$ and so $x \in \comp{A} \cap \comp{B}$. Therefore $\comp{A \cup B} \subseteq \comp{A} \cap \comp{B}$. Now take $x \in \comp{A} \cap \comp{B}$. Then $x \in \comp{A}$ and $x \in \comp{B}$ and so $x \notin A$ and $x \notin B$. Then $x \notin A \cup B$ and so $x \in \comp{A \cup B}$. Therefore $\comp{A} \cap \comp{B} \subseteq \comp{A \cup B}$ and so $\comp{A \cup B} = \comp{A} \cap \comp{B}$.
\renewcommand{\labelenumi}{\arabic{enumi}'.}
\setcounter{enumi}{12}
\item Take $x \in \comp{A \cap B}$. Then $x \notin A \cap B$. Then $x \notin A$ or $x \notin B$. Then $x \in \comp{A}$ or $x \in \comp{B}$ and so $x \in \comp{A} \cup \comp{B}$. Therefore $\comp{A \cap B} \subseteq \comp{A} \cup \comp{B}$. Now take $x \in \comp{A} \cup \comp{B}$. Then $x \in \comp{A}$ or $x \in \comp{B}$ and so $x \notin A$ or $x \notin B$. Then $x \notin A \cap B$ and so $x \in \comp{A \cap B}$. Therefore $\comp{A} \cup \comp{B} \subseteq \comp{A \cap B}$ and so $\comp{A \cap B} = \comp{A} \cup \comp{B}$.
\end{enumerate}

\renewcommand{\labelenumi}{(\Roman{enumi})}
\thm{Theorem 5.3}: The following statements about sets $A$ and $B$ are equivalent to one another.
\env{enumerate}
{\item $A \subseteq B$
\item $A \cap B = A$
\item $A \cup B = B$}

\thm{Proof:}\\
(I) implies (II). Assume $A \subseteq B$. Since, for all $A$ and $B$, $A \cap B \subseteq A$, it is sufficient to prove that $A \subseteq A \cap B$. But if $x \in A$, then $x \in B$ and, hence, $x \in A \cap B$. Hence $A \subseteq A \cap B$.\\
(II) implies (III). Assume $A \cap B = A$. Then $A \cup B = (A \cap B) \cup B = (A \cup B) \cap (B \cup B) = (A \cup B) \cap B = B$.\\
(III) implies (I). Assume $A \cup B = B$. Then this and the identity $A \subseteq A \cup B$ imply $A \subseteq B$.\\

\thm{Note}: The principle of duality does not apply directly to expressions in which $-$ or $\subseteq$ appears. Replace $A - B$ with $A \cap \comp{B}$. Replace $A \subseteq B$ with $A \cap B = A$ or $A \cup B = B$. The dual of $A \cap B = A$ is $A \cup B = A \Leftrightarrow A \supseteq B$. So we can extend the principle of duality to include the inclusion symbol: swap $\subseteq$ with $\supseteq$ (inclusion signs are reversed).\\

\thm{Theory of equations for the algebra of sets}: For an equation formed using $\cup$, $\cap$, and $\comp{\phantom{A}}$ on symbols $A_1, A_2, \dots, A_n$ and $X$ where the $A$'s denote fixed subsets of some universal set $U$ and $X$ denotes a subset of $U$ which is constrained only by the equation in which it appears, determine under what conditions such an equation has a solution and then, assuming these are satisfied, obtain all solutions.\\
Step I. Two sets are equal iff their symmetric difference is equal to $\emptyset$. Hence, an equation in $X$ is equivalent to one whose righthand side is $\emptyset$.\\
Step II. An equation in $X$ with righthand side $\emptyset$ is equivalent to one of the form $$(A \cap X) \cup (B \cap \comp{X}) = \emptyset,$$ where $A$ and $B$ are free of $X$.\\
Step III. The union of two sets is equal to $\emptyset$ iff each set is equal to $\emptyset$. Hence the equation in Step II is equivalent to the pair of simultaneous equations $$A \cap X = \emptyset, B \cap \comp{X} = \emptyset.$$
Step IV. The above pair of equations, and hence the original equation, has a solution iff $B \subseteq \comp{A}$. In this event, any $X$, such that $B \subseteq X \subseteq \comp{A}$, is a solution. [See exercise 5.7]

\section{Relations}
\defn{ordered pair}: $\tuple{x, y} = \set{\set{x}, \set{x,y}}$.

\thm{Theorem 6.1}: The ordered pair of $x$ and $y$ is uniquely determined by $x$ and $y$. Moreover, if $\tuple{x,y} = \tuple{u,v}$ then $x = u$ and $y = v$.

\thm{Proof}:\\
That $x$ and $y$ uniquely determine $\tuple{x,y}$ follows from our assumption that a set is uniquely determined by its members. Now assume $\tuple{x,y} = \tuple{u,v}$. \begin{enumerate}[label=(Case \Roman*)]
\item $u=v$: Then $\tuple{u,v} = \set{\set{u}, \set{u,v}} = \set{\set{u}}$. Hence $\set{\set{x}, \set{x,y}} = \set{\set{u}} \Rightarrow \set{x} = \set{\set{x,y}} = \set{u}$ and so $x=u$ and $y=v$. 
\item $u\neq v$: Then $\set{u} \neq \set{\set{u}, \set{u,v}}$ and $\set{x} \neq \set{\set{u}, \set{u,v}}$. Then $\set{x} \in  \set{\set{u}, \set{u,v}} \Rightarrow \set{x} = \set{u} \Rightarrow x = u$ and $\set{x,y} \in  \set{\set{u}, \set{u,v}} \Rightarrow \set{x,y} = \set{u,v}$. Then $\set{x,y} \neq \set{u}$ and so $x \neq y$ and $y \neq u$. Therefore $y = v$.
\end{enumerate}

\defn{first coordinate}: $x$ in $\tuple{x,y}$.\\
\defn{second coordinate}: $y$ in $\tuple{x,y}$.\\
\defn{ordered triple}: $\tuple{x,y,z} = \tuple{\tuple{x,y},z}$.\\
\defn{ordered $n$-tuple}: $\tuple{x_1, x_2, \dots, x_n} = \tuple{\tuple{x_1, x_2, \dots, x_{n-1}}, x_n}$.\\
\defn{binary relation}: a set of ordered pairs. Given relation $\rho$ and $\tuple{x,y} \in \rho$ we write $x\rho y$.\\
\defn{$\rho$-related}: $x$ is $\rho$-related to $y$ iff $x\rho y$.\\
\defn{$n$-ary relation}: a set of ordered $n$-tuples.\\
\defn{domain}: $D_{\rho} = \set{x \mid \text{for some }y, \tuple{x,y}\in \rho}$.\\
\defn{range}: $R_{\rho} = \set{y \mid \text{for some }x, \tuple{x,y}\in \rho}$.\\
\defn{cartesian product}: $X \times Y = \set{\tuple{x,y} \mid x \in X \wedge y \in Y}$.\\
\defn{relation from $X$ to $Y$}: $\rho \subseteq X \times Y$.\\
\defn{relation in $Z$}: $\rho \subseteq Z \times Z$.\\
\defn{universal relation in $X$}: $\rho = X \times X$.\\
\defn{void relation in $X$}: $\rho = \emptyset$.\\
\defn{identity relation in $X$}: $\iota_{X} = \set{\tuple{x,x} \mid x \in X}$.\\
\defn{$\rho$-relatives of $A$}: $\rho[A]= \set{y \mid x\rho y \text{ for some } x\in A}$. Then we have $\rho(D_{\rho}) = R_{\rho}$, and, for any set $A$, $\rho[A] \subseteq R_{\rho}$.\\

\end{document}