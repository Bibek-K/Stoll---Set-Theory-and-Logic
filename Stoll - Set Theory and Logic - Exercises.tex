\documentclass[11pt]{book}
\parindent=0px
\usepackage{amssymb, amsmath, fullpage, graphicx}
\usepackage{enumitem, siunitx, tikz}
\usepackage[utf8]{inputenc}
\usepackage[T1]{fontenc}

\newcommand{\set}[1]{\{#1\}}
\newcommand{\Set}{\text}
\newcommand{\defn}[1]{\textbf{#1}}
\newcommand{\qed}{\blacksquare}
\newcommand{\comp}{\overline}
\newcommand{\solution}[2]{\item #1\\ \textbf{Solution}: #2}
\newcommand{\thm}[1]{\underline{\textsc{#1}}}
\newcommand{\env}[2]{\begin{#1} #2 \end{#1}}

\begin{document}
\chapter{Sets and Relations}

\renewcommand{\labelenumi}{1.2.\arabic{enumi}}
\begin{enumerate}
\solution{Explain why $2 \in \set{1,2,3}$.}{$2 \in \set{1,2,3}$ by definition: 2 is an element of $\set{1,2,3}$.}
\solution{Is $\set{1,2} \in \set{\set{1,2,3}, \set{1,3},1,2}$? Justify your answer.}{$\set{1,2} \notin \set{\set{1,2,3},\set{1,3},1,2}$ since it does not appear as a member of the given set.}
\solution{Try to devise a set which is a member of itself.}{The set of all sets.}
\solution{Give an example of sets $A, B,$ and $C$ such that $A \in B$, $B \in C$, and $A \notin C$.}{$\Set{A} = \set{1}$. $\Set{B} = \set{1, \set{1}}$. $\Set{C} = \set{1, \set{1, \set{1}}}$. Then $\Set{A} \in \Set{B}$, $\Set{B} \in \Set{C}$, but $\Set{A} \notin \Set{C}$.}
\solution{Describe in prose each of the following sets.} 
	{\begin{enumerate}
	\solution{$\set{x \in \mathbb{Z} \mid x \text{ is divisible by } 2 \text{ and } x \text{ is divisible by } 3}$}{All integer multiples of 6.}
	\solution{$\set{x \mid x \in A \text{ and } x \in B}$}{All elements common to sets A and B.}
	\solution{$\set{x \mid x \in A \text{ or } x \in B}$}{All elements from set A and from set B.}
	\solution{$\set{x \in \mathbb{Z}^+ \mid x \in \set{x \in \mathbb{Z} \mid \text{for some integer } y,\, x = 2y} \text{ and } x \in \set{x \in \mathbb{Z} \mid \text{for some integer } y,\, x = 3y}}$}{All positive integer multiples of 6.}
	\solution{$\set{x^2 \mid x \text{ is a prime}}$}{The squares of all prime numbers.}
	\solution{$\set{a/b \in \mathbb{Q} \mid a + b = 1 \text{ and } a, b \in \mathbb{Q}}$}{All ratios of rational numbers whose numerator and denominator sum to 1; i.e. all rational numbers.}
	\solution{$\set{(x,y) \in \mathbb{R}^2 \mid x^2 + y^2 = 1}$}{All points on the unit circle.}
	\solution{$\set{(x,y) \in \mathbb{R}^2 \mid y = 2x \text{ and } y = 3x}$}{The single point (0,0).}
	\end{enumerate}}
\solution{Prove that if $a,b,c$, and $d$ are any objects, not necessarily distinct from one another, then $\set{\set{a},\set{a,b}} = \set{\set{c},\set{c,d}}$ iff both $a = c$ and $b = d$.}
{($\then$) Let $\set{\set{a},\set{a,b}} = \set{\set{c},\set{c,d}}$. Then $\set{\set{a},\set{a,b}} \subseteq \set{\set{c},\set{c,d}}$ and so in particular $\set{a} \in \set{\set{c},\set{c,d}}$. Suppose $c = d$. Then $\set{\set{c},\set{c,d}} = \set{\set{c},\set{c,c}} = \set{\set{c},\set{c}} = \set{\set{c}}$ and therefore $\set{a} \in \set{\set{c}} \then \set{a} = \set{c} \then \set{a} \subseteq \set{c} \then a \in \set{c} \then a = c$. Then since $\set{\set{a},\set{a,b}} \subseteq \set{\set{c},\set{c,d}} = \set{\set{c}} = \set{\set{a}}$ we also have that $\set{a,b} \in \set{\set{a}} \then \set{a,b} = \set{a} \then \set{a,b} \subseteq \set{a} \then b \in \set{a} \then a = b$. Now suppose $c \neq d$. Since $\set{a} \in \set{\set{c},\set{c,d}} \then \set{a} = \set{c} \then a = c$. By $\set{\set{a},\set{a,b}} \subseteq \set{\set{c},\set{c,d}}$ we also have that $\set{a,b} = \set{c,b} \in \set{\set{c},\set{c,d}}$ and so $\set{c,b} = \set{c,d} \then b = d$.\\ ($\Leftarrow$) Let $a = c$ and $b = d$. Then $\set{\set{a},\set{a,b}} = \set{\set{c},\set{c,d}}$. $\qed$}
\end{enumerate}

\hrulefill

\renewcommand{\labelenumi}{1.3.\arabic{enumi}}
\begin{enumerate}
\item Prove each of the following, using any properties of numbers that may be needed.
	\begin{enumerate}
	\solution{$\set{x \in \mathbb{Z} \mid \text{for an integer } y,\, x = 6y} = \set{x \in \mathbb{Z} \mid \text{for integers } u \text{ and } v,\, x = 2u \text{ and } x = 3v}$.}
	{Let $a \in \set{x \in \mathbb{Z} \mid \text{for an integer } y,\, x = 6y}$. Then $a = 6b$ for an integer $b$. Then $a = 2\cdot 3b$ and $a = 3 \cdot 2b$. Since $b$ is an integer, $3b$ and $2b$ are also integers. Therefore $a \in \set{x \in \mathbb{Z} \mid \text{for integers } u \text{ and } v,\, x = 2u \text{ and } x = 3v}$ and so $\set{x \in \mathbb{Z} \mid \text{for an integer } y,\, x = 6y} \subseteq \set{x \in \mathbb{Z} \mid \text{for integers } u \text{ and } v,\, x = 2u \text{ and } x = 3v}$.\\ Now let $a \in \set{x \in \mathbb{Z} \mid \text{for integers } u \text{ and } v,\, x = 2u \text{ and } x = 3v}$. Then there is an $m$ and $n$ such that $a = 2m$ and $a = 3n$. Then $m - n = \frac{a}{6}$ and so $a = 6(m - n)$. Since $m$ and $n$ are integers, $m - n$ is also an integer and so $a \in \set{x \in \mathbb{Z} \mid \text{for an integer } y,\, x = 6y}$. Therefore $\set{x \in \mathbb{Z} \mid \text{for integers } u \text{ and } v,\, x = 2u \text{ and } x = 3v} \subseteq \set{x \in \mathbb{Z} \mid \text{for an integer } y,\, x = 6y}$.\\ Thus $\set{x \in \mathbb{Z} \mid \text{for an integer } y,\, x = 6y} = \set{x \in \mathbb{Z} \mid \text{for integers } u \text{ and } v,\, x = 2u \text{ and } x = 3v}$. $\qed$}
	\solution{$\set{x \in \mathbb{R} \mid \text{for a real number } y,\, x = y^2} = \set{x \in \mathbb{R} \mid x \geq 0}$}
	{Let $a \in \set{x \in \mathbb{R} \mid \text{for a real number } y,\, x = y^2}$. Then $a = b^2$ for some $b \in \mathbb{R}$. If $b = 0$ then $a = 0$. Otherwise, $a > 0$ since the square of a nonzero real number is positive. Thus $a \in \set{x \in \mathbb{R} \mid x \geq 0}$ and therefore $\set{x \in \mathbb{R} \mid \text{for a real number } y,\, x = y^2} \subseteq \set{x \in \mathbb{R} \mid x \geq 0}$.\\ Now let $a \in \set{x \in \mathbb{R} \mid x \geq 0}$. Then $a \geq 0$. Then we may find a $b \in \mathbb{R}$ such that $a = b^2$: simply pick $b = \sqrt{a} \in \mathbb{R}$. (The square root of any nonnegative real number is again a real number.) Then $a \in \set{x \in \mathbb{R} \mid \text{for a real number } y,\, x = y^2}$ and therefore $\set{x \in \mathbb{R} \mid x \geq 0} \subseteq \set{x \in \mathbb{R} \mid \text{for a real number } y,\, x = y^2}$.\\ Thus $\set{x \in \mathbb{R} \mid \text{for a real number } y,\, x = y^2} = \set{x \in \mathbb{R} \mid x \geq 0}$. $\qed$}
	\solution{$\set{x \in \mathbb{Z} \mid \text{for an integer }y,\, x = 6y} \subseteq \set{x \in \mathbb{Z} \mid \text{for an integer }y,\, x = 2y}$.}
	{Let $a \in \set{x \in \mathbb{Z} \mid \text{for an integer }y,\, x = 6y}$. Then $a = 6b$ for some integer $b$. Then $a = 2 \cdot 3b$. Since $b$ is an integer, $3b$ is also an integer. Therefore $a \in \set{x \in \mathbb{Z} \mid \text{for an integer }y,\, x = 2y}$. Thus $\set{x \in \mathbb{Z} \mid \text{for an integer }y,\, x = 6y} \subseteq \set{x \in \mathbb{Z} \mid \text{for an integer }y,\, x = 2y}$. $\qed$}
	\end{enumerate}
\item Prove each of the following for sets $A, B$, and $C$.
	\begin{enumerate}
	\solution{If $A \subseteq B$ and $B \subseteq C$, then $A \subseteq C$.}
	{Suppose $A \subseteq B$ and $B \subseteq C$. Then for any $a \in A$ we have that $a \in B$ and since $a \in B$ we have that $a \in C$. Thus $a \in C$ whenever $a \in A$ and therefore $A \subseteq C$.}
	\solution{If $A \subseteq B$ and $B \subset C$, then $A \subset C$.}
	{Suppose $A \subseteq B$ and $B \subset C$. Take the case when $A = B$. Since $B \subset C$ we have that $A \subset C$. Now take the case when $A \subset B$. For any $a \in A$ we have that $a \in B$. Since $B \subset C$ we have that $a \in C$. Then $A \subseteq C$. Since $B \subset C$, we know that there is some element $c \in C$ such that $c \notin B$ and since $A \subset B$ we know that $c \notin A$ and so $A \neq C$. Therefore $A \subset C$.}
	\solution{If $A \subset B$ and $B \subseteq C$, then $A \subset C$.}
	{Suppose $A \subset B$ and $B \subseteq C$. Take the case when $B = C$. Since $A \subset B$ we have that $A \subset C$. Now take the case when $B \subset C$. For any $a \in A$ we have that $a \in B$. Since $B \subset C$ we have that $a \in C$. Then $A \subseteq C$. Since $B \subset C$, we also know that there is some element $c \in C$ such that $c \notin B$ and since $A \subset B$ we know that $c \notin A$ and so $A \neq C$. Therefore $A \subset C$.}
	\solution{If $A \subset B$ and $B \subset C$, then $A \subset C$.}
	{Suppose $A \subset B$ and $B \subset C$. For any $a \in A$ we have that $a \in B$. Since $B \subset C$ we have that $a \in C$. Then $A \subset C$. Since $B \subset C$, we know that there is some element $c \in C$ such that $c \notin B$ and since $A \subset B$ we know that $c \notin A$ and so $A \neq C$. Therefore $A \subset C$.}
	\end{enumerate}
\solution{Give an example of sets $A,B,C,D$, and $E$ which satisfy the following conditions simultaneously: $A \subset B$, $B \in C$, $C \subset D$, and $D \subset E$.}
{Let $A = \emptyset$, $B = \set{\emptyset}$, $C = \set{\set{\emptyset}}$, $D = \set{\emptyset, \set{\emptyset}}$, and $E = \set{\emptyset, \set{\emptyset}, \set{\set{\emptyset}}}$.}
\item Which of the following are true for all sets $A, B$, and $C$?
	\begin{enumerate}
	\solution{If $A \notin B$ and $B \notin C$, then $A \notin C$.}
	{False: Let $A = \emptyset$, $B = \set{0}$ and $C = \set{\emptyset}$.}
	\solution{If $A \neq B$ and $B \neq C$, then $A \neq C$.}
	{False: Let $A = \mathbb{R}$, $B = \mathbb{Z}$ and $C = \mathbb{R}$.}
	\solution{If $A \in B$ and $B \not\subseteq C$, then $A \notin C$.}
	{False: Let $A = \emptyset$, $B = \set{\emptyset, 0}$ and $C = \set{\emptyset, 1}$.}
	\solution{If $A \subset B$ and $B \subseteq C$, then $C \not\subseteq A$.}
	{True: Suppose $A \subset B$ and $B \subseteq C$. Since $A \subset B$, there is some $b \in B$ such that $b \notin A$. Since $B \subseteq C$ we have that this $b \in C$ and thus there is a $c \in C$ such that $c \notin A$. Therefore $C \not\subseteq A$.}
	\solution{If $A \subset B$ and $B \subset C$, then $A \subset C$.}
	{True: Suppose $A \subset B$ and $B \subset C$. Then for any $a \in A$ we have that $a \in B$. Since $a \in B$ we have that $a \in C$. Therefore $A \subseteq C$. Since $B \subset C$ we have that there is some $c \in C$ such that $c \notin B$. Since $A \subset B$ we have that $c \notin A$. Therefore $A \neq C$ and so $A \subset C$.}
	\end{enumerate}
\solution{Show that for every set $A$, $A \subseteq \emptyset$ iff $A = \emptyset$.}
{($\then$) Suppose $A \subseteq \emptyset$. Then for any $a \in A$ we have that $a \in \emptyset$. But since the empty set has no members, no such $a$ can exist. Therefore $A$ has no members and so $A = \emptyset$. ($\Leftarrow$) Suppose $A = \emptyset$. Then $A$ has no members and so, certainly, for all $a \in A$ we have that $a \in B$, for any set $B$. Therefore $A \subseteq B$. Letting $B = \emptyset$, we conclude that $A \subseteq \emptyset$. $\qed$}
\solution{Let $A_1,A_2,\dots,A_n$ be $n$ sets. Show that $$A_1 \subseteq A_2 \subseteq \dots \subseteq A_n \subseteq A_1 \text{ iff } A_1 = A_2 = \dots = A_n.$$}
{($\Leftarrow$) Suppose sets $A_1 = A_2 = \dots = A_n$. Then clearly $A_i \subseteq A_{i+1}$ for all $1 \leq i \leq n - 1$ and $A_n \subseteq A_1$. Therefore $A_1 \subseteq A_2 \subseteq \dots \subseteq A_n \subseteq A_1$.\\ ($\then$) Suppose $A_1 \subseteq A_2 \subseteq \dots \subseteq A_n \subseteq A_1$. Then for any $a \in A_1$ we have that $a \in A_2$, $a \in A_3$, $\dots$, $a \in A_n$ and so $A_1 \subseteq A_n$. But since $A_n \subseteq A_1$ we must have that $A_1 = A_n$. Therefore $A_1 \subseteq A_2 \subseteq \dots \subseteq A_{n-1} \subseteq A_1$. Repeating this argument $n - 2$ times for $j = n - 1, n - 2, \dots, 2$ we find that for any $a \in A_1$ we have that $a \in A_j$ and therefore $A_1 \subseteq A_j$. But we also have that $A_j \subseteq A_1$ and so $A_1 = A_j$. Finally, we conclude that $A_1 = A_2 = \dots = A_n$. $\qed$}
\solution{Give several examples of a set $X$ such that each element of $X$ is a subset of $X$.}
{$X_1 = \emptyset$. $X_2 = \set{\emptyset}$. $X_3 = ?$}
\solution{List the members of $\mathcal{P}(A)$ if $A = \set{\set{1,2}, \set{3}, 1}$.}
{$\mathcal{P}(A) = \set{A, \set{\set{1,2}, \set{3}}, \set{\set{1,2}, 1}, \set{\set{3}, 1}, \set{\set{1,2}}, \set{\set{3}}, \set{1}, \emptyset}$}
\solution{For each positive integer $n$, give an example of a set $A_n$ of $n$ elements such that for each pair of elements of $A_n$, one member is an element of the other.}
{Let $A_0 = \emptyset$ and $A_1 = \set{A_0}$. For $n \geq 2$, let $A_n = \set{A_0, A_1, \dots, A_{n-1}}$.}
\end{enumerate}

\hrulefill

\renewcommand{\labelenumi}{1.4.\arabic{enumi}}
\begin{enumerate}
\solution{Prove that for all sets $A$ and $B$, $\emptyset \subseteq A \cap B \subseteq A \cup B$.}
{Let $A$ and $B$ be any sets. Since the empty set has no members, it's clear that for all $x \in \emptyset$ we have that $x \in C$ for any set $C$. Since $A \cap B$ is a set, we have that $\emptyset \subseteq A \cap B$. Now take any $x \in A \cap B$. Then $x \in A$ and $x \in B$ and so, clearly, $x \in A$ or $x \in B$. Therefore $x \in A \cup B$ and so $A \cap B \subseteq A \cup B$. Thus $\emptyset \subseteq A \cap B \subseteq A \cup B$.}
\solution{Let $\mathbb{Z}$ be the universal set, and let
\begin{align*}
A &= \set{x \in \mathbb{Z} \mid \text{for some positive integer } y,\, x = 2y},\\
B &= \set{x \in \mathbb{Z} \mid \text{for some positive integer } y,\, x = 2y - 1},\\
C &= \set{x \in \mathbb{Z} \mid x < 10}.
\end{align*}
Describe $\comp{A}, \comp{A \cup B}, A - \comp{C}$, and $C - (A \cup B)$, either in prose or by a defining property.}
{\begin{itemize}
\item $\comp{A} = \set{x \in \mathbb{Z} \mid x < 1} \cup B$
\item $\comp{A \cup B} = \set{x \in \mathbb{Z} \mid x < 1}$
\item $A - \comp{C} = \set{2,3,6,8}$
\item $C - (A \cup B) = \set{x \in \mathbb{Z} \mid x < 1}$
\end{itemize}}
\item Consider the following subsets of $\mathbb{Z^+}$, the set of positive integers:
\begin{align*}
A &= \set{x \in \mathbb{Z^+} \mid \text{for some integer } y,\, x = 2y},\\
B &= \set{x \in \mathbb{Z^+} \mid \text{for some integer } y,\, x = 2y + 1},\\
C &= \set{x \in \mathbb{Z^+} \mid \text{for some integer } y,\, x = 3y}.
\end{align*}
	\begin{enumerate}
	\solution{Describe $A \cap C, B \cup C$, and $B - C$.}
	{\begin{itemize}
		\item $A \cap C = \set{x \in \mathbb{Z^+} \mid \text{for some integer } y,\, x = 6y}$. This is the set of all positive multiples of both 2 and 3 (i.e. all positive multiples of 6).
		\item $B \cup C = B \cup \set{x \in \mathbb{Z^+} \mid \text{for some integer } y,\, x = 6y}$. This is the set of all positive odd integers along with all even positive multiples of 3 (i.e. all positive multiples of 6).
		\item $B - C = \set{x \in \mathbb{Z^+} \mid \text{for some integer } y,\, x = 3y + 1 \text{ or } x = 3y + 2}$. This is the set of all positive integers which are not divisible by 3.
	 \end{itemize}}
	\solution{Verify that $A \cap (B \cup C) = (A \cap B) \cup (A \cap C)$.}
	{In this example:
	\begin{itemize}
		\item $A \cap (B \cup C) = A \cap C$
		\item $A \cap B = \emptyset \then (A \cap B) \cup (A \cap C) = A \cap C$
	\end{itemize}
	A general proof:\\
	Assume $x \in A \cap (B \cup C)$. Then by definition of set intersection, $x \in A$ and $x \in B \cup C$. By the definition of set union, $x \in B$ or $x \in C$. If $x \in B$ then we have that $x \in A \cap B$ since $x \in A$. Otherwise, if $x \in C$ then we have that $x \in A \cap C$ since $x \in A$. In both cases we know that $x \in (A \cap B) \cup (A \cap C)$ since both $A \cap B \subseteq (A \cap B) \cup (A \cap C)$ and $A \cap C \subseteq (A \cap B) \cup (A \cap C)$. Therefore $A \cap (B \cup C) \subseteq (A \cap B) \cup (A \cap C)$. Now assume $x \in (A \cap B) \cup (A \cap C)$. By the definition of set union, $x \in A \cap B$ or $x \in A \cap C$. If $x \in A \cap B$ then by the definition of set intersection, we have that $x \in A$ and $x \in B$. Since $B \subseteq B \cup C$ we know that $x \in B \cup C$. Thus $x \in A \cap (B \cup C)$. Otherwise if $x \in A \cap C$ then by the definition of set intersection, we have that $x \in A$ and $x \in C$. Since $C \subseteq B \cup C$ we know that $x \in B \cup C$. Thus $x \in A \cap (B \cup C)$. Therefore $(A \cap B) \cup (A \cap C) \subseteq  A \cap (B \cup C)$. Therefore $(A \cap B) \cup (A \cap C) =  A \cap (B \cup C)$. $\qed$}
	\end{enumerate}
\solution{If $A$ is any set, what are each of the following sets? $A \cap \emptyset$, $A \cup \emptyset$, $A - \emptyset$, $A - A$, $\emptyset - A$.}
{\begin{itemize}
\item $A \cap \emptyset = \set{x \mid x \in A \text{ and } x \in \emptyset} = \emptyset$.
\item $A \cup \emptyset = \set{x \mid x \in A \text{ or } x \in \emptyset} = A$.
\item $A - \emptyset = \set{x \mid x \in A \text{ and } x \notin \emptyset} = A$.
\item $A - A = \set{x \mid x \in A \text{ and } x \notin A} = \emptyset$.
\item $\emptyset - A = \set{x \mid x \in \emptyset \text{ and } x \notin A} = \emptyset$.
\end{itemize}}
\solution{Determine $\emptyset \cap \set{\emptyset}, \set{\emptyset} \cap \set{\emptyset}, \set{\emptyset, \set{\emptyset}} - \emptyset, \set{\emptyset, \set{\emptyset}} - \set{\emptyset}, \set{\emptyset,\set{\emptyset}} - \set{\set{\emptyset}}$}
{\begin{itemize}
\item $\emptyset \cap \set{\emptyset} = \set{x \mid x \in \emptyset \text{ and } x \in \set{\emptyset}} = \emptyset$.
\item $\set{\emptyset} \cap \set{\emptyset} = \set{x \mid x \in \set{\emptyset} \text{ and } x \in \set{\emptyset}} = \set{x \mid x \in \set{\emptyset}} = \set{\emptyset}$.
\item $\set{\emptyset, \set{\emptyset}} - \emptyset = \set{x \mid x \in \set{\emptyset, \set{\emptyset}} \text{ and } x \notin \emptyset} = \set{\emptyset, \set{\emptyset}}$.
\item $\set{\emptyset, \set{\emptyset}} - \set{\emptyset} = \set{x \mid x \in \set{\emptyset, \set{\emptyset}} \text{ and } x \notin \set{\emptyset}} = \set{\set{\emptyset}}$.
\item $\set{\emptyset,\set{\emptyset}} - \set{\set{\emptyset}} = \set{x \mid x \in \set{\emptyset,\set{\emptyset}} \text{ and } x \notin \set{\set{\emptyset}}} = \set{\emptyset}$.
\end{itemize}}
\solution{Suppose $A$ and $B$ are subsets of $U$. Show that in each of (a), (b), and (c) below, if any one of the relations stated holds, then each of the others holds.
\begin{enumerate}
\item $A \subseteq B, \comp{A} \supseteq \comp{B}, A \cup B = B, A \cap B = A$.
\item $A \cap B = \emptyset, A \subseteq \comp{B}, B \subseteq \comp{A}$.
\item $A \cup B = U, \comp{A} \subseteq B, \comp{B} \subseteq A$.}
\end{enumerate}
{\begin{enumerate}
\item \begin{enumerate}
		\item Assume $A \subseteq B$. 
		\begin{itemize}
			\item Take $x \notin B$. Assume $x \in A$. Since $A \subseteq B$ we have that $x \in B$. Then by contradiction we must have that $x \notin A$. Therefore $\comp{B} \subseteq \comp{A}$ or $\comp{A} \supseteq \comp{B}$.
			\item Take $x \in A \cup B$. Then $x \in A$ or $x \in B$. Assume $x \in A$. Then since $A \subseteq B$ we have that $x \in B$. Therefore $A \cup B \subseteq B$. But since $B \subseteq A \cup B$ we have that $A \cup B = B$.
			\item Take $x \in A$. Because $A \subseteq B$ we have that $x \in B$. Since $x \in A$ and $x \in B$ we have that $x \in A \cap B$ and therefore $A \subseteq A \cap B$. But since $A \cap B \subseteq A$, we have $A \cap B = A$.
		\end{itemize}
		\item Assume $\comp{A} \supseteq \comp{B}$.
		\begin{itemize}
		\item Take $x \in A$. Assume $x \notin B$. Then because $\comp{A} \supseteq \comp{B}$ we have that $x \notin A$. Then by contradiction we must have that $x \in B$. Therefore $A \subseteq B$.
		\item Take $x \in A \cup B$. Then by the definition of set union, $x \in A$ or $x \in B$. Assume $x \notin B$. Then because $\comp{A} \supseteq \comp{B}$ we have that $x \notin A$. But then since $x \notin A$ and $x \notin B$ we have that $x \notin A \cup B$. By contradiction we must have that $x \in B$ and thus $A \cup B \subseteq B$. But since $B \subseteq A \cup B$ we have $A \cup B = B$.
		\item Take $x \in A$. Assume $x \notin B$. Then because $\comp{A} \supseteq \comp{B}$ we have that $x \notin A$. By contradiction, $x \in B$. Thus $A \subseteq A \cap B$. But since $A \cap B \subseteq A$ we have $A \cap B = A$.
		\end{itemize}
		\item Assume $A \cup B = B$.
		\begin{itemize}
		\item Take $x \in A$. Then $x \in A \cup B = B$ and therefore $A \subseteq B$.
		\item Take $x \notin B$. Then $x \notin A \cup B$ and so $x \notin A$. Therefore $\comp{A} \supseteq \comp{B}$.
		\item Take $x \in A$. Then $x \in A \cup B = B$. Since $x \in A$ and $x \in B$ then $x \in A \cap B$ and so $A \subseteq A \cap B$. But since $A \cap B \subseteq A$ we have $A \cap B = A$.
		\end{itemize}
		\item Assume $A \cap B = A$.
		\begin{itemize}
		\item Take $x \in A$. Then $x \in A \cap B$ and so $x \in B$. Therefore $A \subseteq B$.
		\item Take $x \notin B$. Then $x \notin A \cap B$ and so $x \notin A$. Therefore $\comp{A} \supseteq \comp{B}$.
		\item Take $x \in A \cup B$. Then $x \in A$ or $x \in B$. Assume $x \notin B$. Then $x \notin A \cap B = A$ and so $x \notin A \cup B$. By contradiction, we must have the $x \in B$. Then $A \cup B \subseteq B$. But since $B \subseteq A \cup B$ we have that $A \cup B = B$.
		\end{itemize}
		\end{enumerate}
\item \begin{enumerate}
		\item Assume $A \cap B = \emptyset$.
		\begin{itemize}
			\item Take $x \in A$. Since $A \cap B = \emptyset$, $x \notin B$. Then $A \subseteq \comp{B}$.
			\item Take $x \in B$. Since $A \cap B = \emptyset$, $x \notin A$. Then $B \subseteq \comp{A}$.
		\end{itemize}
		\item Assume $A \subseteq \comp{B}$.
		\begin{itemize}
			\item Take $x \in A$. Then since $A \subseteq \comp{B}$ we have $x \notin B$. Now take $x \in B$. Assume $x \in A$. But then since $A \subseteq \comp{B}$ we have $x \notin B$. By contradiction we must have that $x \notin A$. Thus $x \in A \then x \notin B$ and $x \in B \then x \notin A$ and so $A \cap B = \emptyset$.
			\item Take $x \in B$. Assume $x \in A$. Then since $A \subseteq \comp{B}$ we have that $x \notin B$. Then by contradiction, we must have $x \notin A$ and so $B \subseteq \comp{A}$.
		\end{itemize}
		\item Assume $B \subseteq \comp{A}$.
		\begin{itemize}
			\item Take $x \in B$. Then since $B \subseteq \comp{A}$ we have $x \notin A$. Now take $x \in A$ and assume $x \in B$. But since $B \subseteq \comp{A}$ we have $x \notin A$. Then by contradiction, we must have that $x \notin B$. Thus $x \in A \then x \notin B$ and $x \in B \then x \notin A$ and so $A \cap B = \emptyset$.
			\item Take $x \in A$. Assume $x \in B$. Then since $B \subseteq \comp{A}$ we have $x \notin A$. Then by contradiction we must have that $x \notin B$. Therefore $A \subseteq \comp{B}$.
		\end{itemize}
	  \end{enumerate}
\item \begin{enumerate}
		\item Assume $A \cup B = U$.
		\begin{itemize}
			\item Take $x \notin A$. Assume $x \notin B$. Then since $x \notin A$ and $x \notin B$, we have that $x \notin A \cup B = U$. But since $A \subseteq U$ this is a contradiction. So we must have that $x \in B$. Therefore $\comp{A} \subseteq B$.
			\item Take $x \notin B$. Assume $x \notin A$. Then since $x \notin A$ and $x \notin B$, we have that $x \notin A \cup B = U$. But since $B \subseteq U$ this is a contradiction. So we must have that $x \in A$. Therefore $\comp{B} \subseteq A$.
		\end{itemize}
		\item Assume $\comp{A} \subseteq B$.
		\begin{itemize}
			\item Take $x \in U$. Then either $x \in A$ or $x \notin A$. Assume that $x \in A$. Then $x \in A \cup B$. Now assume that $x \notin A$. Then since $\comp{A} \subseteq B$ we have that $x \in B$. Then $x \in A \cup B$. Therefore $U \subseteq A \cup B$. But since $A \cup B \subseteq U$ we must have that $A \cup B = U$. 
			\item Take $x \notin B$. Assume$ \comp{B} \subseteq A$
		\end{itemize}
		\item Assume $\comp{B} \subseteq A$.
		\begin{itemize}
			\item Take $x \in U$. Then $x \in B$ or $x \notin B$. If $x \in B$ then $x \in A \cup B$. If $x \notin B$ then since $\comp{B} \subseteq A$ we have that $x \in A$ and so $x \in A \cup B$. Therefore $U \subseteq A \cup B$. But since $A \cup B \subseteq U$ we must have that $A \cup B = U$.
			\item Take $x \notin A$ and assume $x \notin B$. Then since $\comp{B} \subseteq A$ we have that $x \in A$. Then by contradiction we must have that $x \in B$ and so $\comp{A} \subseteq B$.
		\end{itemize}
	\end{enumerate}
\end{enumerate}}
\solution{Prove that for all sets $A,B$, and $C$, $$(A \cap B) \cup C = A \cap (B \cup C) \text{ iff } C \subseteq A.$$}
{($\then$) Assume $(A \cap B) \cup C = A \cap (B \cup C)$. Let $x \in C$. Then $x \in (A \cap B) \cup C$. Since $(A \cap B) \cup C = A \cap (B \cup C)$ we have that $x \in A \cap (B \cup C)$ and thus $x \in A$. Therefore $C \subseteq A$.\\ ($\Leftarrow$) Assume $C \subseteq A$. Let $x \in (A \cap B) \cup C$. Then $x \in A \cap B$ or $x \in C$. Assume $x \in A \cap B$. Then $x \in A$ and $x \in B$. Since $x \in B$ then $x \in B \cup C$. Since $x \in A$ and $x \in B \cup C$ we have that $x \in A \cap (B \cup C)$. Now assume $x \in C$. Then $x \in B \cup C$. Since $C \subseteq A$ we also have that $x \in A$. Thus $x \in A \cap (B \cup C)$. Therefore $(A \cap B) \cup C \subseteq A \cap (B \cup C)$. Now let $x \in A \cap (B \cup C)$. Then $x \in A$ and $x \in B \cup C$. Then $x \in B$ or $x \in C$. Assume $x \in B$. Then since $x \in A$ and $x \in B$ we have that $x \in A \cap B$. Therefore $x \in (A \cap B) \cup C$. Now assume $x \in C$. Then $x \in (A \cap B) \cup C$. Therefore $A \cap (B \cup C) \subseteq (A \cap B) \cup C$ and so we can conclude that $(A \cap B) \cup C = A \cap (B \cup C)$. $\qed$}
\solution{Prove that for all sets $A,B$, and $C$, $$(A - B) - C = (A - C) - (B - C).$$}
{Let $x \in (A - B) - C$. Then $x \in A - B$ and $x \notin C$. Since $x \in A - B$ we have that $x \in A$ and $x \notin B$. Then since $x \in A$ and $x \notin C$ we have that $x \in A - C$. Since $x \notin B$ we have that $x \notin B - C$. Then since $x \in A - C$ and $x \notin B - C$ we have that $x \in (A - C) - (B - C)$. Now let $x \in (A - C) - (B - C)$. Then $x \in A - C$ and $x \notin B - C$. Then since $x \in A - C$ we have that $x \in A$ and $x \notin C$. Since $x \notin B - C$ we have that either $x \notin B$ or $x \in C$. But $x \in C$ is a contradiction because $x \notin C$. Therefore $x \notin B$. Then since $x \in A$ and $x \notin B$ we have that $x \in A - B$ and since $x \notin C$ we have that $x \in (A - B) - C$. Therefore $(A - C) - (B - C) \subseteq (A - B) - C$ and thus $(A - B) - C = (A - C) - (B - C)$. $\qed$}
\item{\begin{enumerate}
\solution{Draw the Venn diagram of the symmetric difference, $A + B$, of sets $A$ and $B$.}
{... TO DO ...}
\solution{Using a Venn diagram, show that the symmetric difference is a commutative and associative operation.}
{... TO DO ...}
\solution{Show that for every set $A$, $A + A = \emptyset$ and $A + \emptyset = A$}
{Let $A$ be a set. Let $x \in A + A = (A - A) \cup (A - A)$. Then $x \in A - A = \set{x \mid x \in A \text{ and } x \notin A} = \emptyset$. Therefore $A + A \subseteq \emptyset$. Then since $\emptyset \subseteq A + A$ we have that $A + A = \emptyset$. $\qed$\\ Let $x \in A + \emptyset = (A - \emptyset) \cup (\emptyset - A)$. Then $x \in A - \emptyset$ or $x \in \emptyset - A$. Assume $x \in \emptyset - A$. Then $x \in \emptyset$. But this is a contradiction and so we have that $x \in A - \emptyset$. Then $x \in A$. Therefore $A + \emptyset \subseteq A$. Now let $x \in A$. Since $x \notin \emptyset$ we have that $x \in A - \emptyset$. Then $x \in (A - \emptyset) \cup (\emptyset - A) = A + \emptyset$. Thus $A \subseteq A + \emptyset$ and therefore $A + \emptyset = A$. $\qed$}
\end{enumerate}}
\solution{The Venn diagram for subsets $A, B$, and $C$ of $U$, in general, divides the rectangle representing $U$ into eight nonoverlapping regions. Label each region with a combination of $A, B$, and $C$ which represents exactly that region.}
{... TO DO ...}
\item{With the aid of a Venn diagram investigate the validity of each of the following inferences:
\begin{enumerate}
\solution{If $A, B$, and $C$ are subsets of $U$ such that $A \cap B \subseteq \comp{C}$ and $A \cup C \subseteq B$, then $A \cap C = \emptyset$.}
{... TO DO ...}
\solution{If $A, B$, and $C$ are subsets of $U$ such that $A \subseteq \comp{B \cup C}$ and $B \subseteq \comp{A \cup C}$, then $B = \emptyset$.}
{... TO DO ...}
\end{enumerate}}
\end{enumerate}

\hrulefill

\renewcommand{\labelenumi}{1.5.\arabic{enumi}}
\begin{enumerate}
\solution{Prove that parts 3', 4', and 5' of Theorem 5.1 are identities}
{\renewcommand{\labelenumii}{\arabic{enumii}'.}
\begin{enumerate}
\setcounter{enumii}{2}
\item Assume $x \in A \cap (B \cup C)$. Then $x \in A$ and $x \in B \cup C$. If $x \in B$ then since $x \in A$ we have $x \in A \cap B$ and so $x \in (A \cap B) \cup (A \cap C)$. Otherwise if $x \in C$ then since $x \in A$ we have $x \in A \cap C$ and so $x \in (A \cap B) \cup (A \cap C)$. Therefore $A \cap (B \cup C) \subseteq (A \cap B) \cup (A \cap C)$. Now assume $x \in (A \cap B) \cup (A \cap C)$. Then $x \in A \cap B$ or $x \in A \cap C$. If $x \in A \cap B$ then $x \in A$ and $x \in B$. Since $x \in B$ we have $x \in B \cup C$. Since we also have $x \in A$ then $x \in A \cap (B \cup C)$. Otherwise if $x \in A \cap C$ then $x \in A$ and $x \in C$. Since $x \in C$ we have $x \in B \cup C$. Since we also have $x \in A$ then $x \in A \cap (B \cup C)$. Therefore $(A \cap B) \cup (A \cap C) \subseteq A \cap (B \cup C)$. Hence $A \cap (B \cup C) = (A \cap B) \cup (A \cap C)$.\\
\item Assume $x \in A \cap U$. Then $x \in A$ and $x \in U$. Therefore $A \cap U \subseteq A$. Now assume $x \in A$. Then since $A \subseteq U$ we have $x \in U$ and so $x \in A \cap U$. Therefore $A \subseteq A \cap U$. Hence $A \cap U = A$.\\
\item Assume $x \in A \cap \comp{A}$. Then $x \in A$ and $x \in \comp{A}$. Since $x \in \comp{A}$ we have $x \notin A$. Since $x \in A$ and $x \notin A$ we have $x \in \emptyset$. Therefore $A \cap \comp{A} \subseteq \emptyset$. Since $\emptyset \subseteq X$ for any set $X$ we have $\emptyset \subseteq A \cap \comp{A}$. Hence $A \cap \comp{A} = \emptyset$.
\end{enumerate}}
\solution{Prove the unprimed parts of Theorem 5.2 using the membership relations. Try to prove the same results using only Theorem 5.1. In at least one such proof write out the dual of each step to demonstrate that a proof of the dual results.}
{\renewcommand{\labelenumii}{\arabic{enumii}.}
\begin{enumerate}
\setcounter{enumii}{5}
\item %6%
\underline{using membership}: Assume that $A \cup B = A$ for all $A$. Then, in particular, for $A = \emptyset$ we have $A \cup B = \emptyset \cup B = \emptyset$. Take $x \in B$. Then $x \in \emptyset \cup B = \emptyset$ and so $B \subseteq \emptyset$. Now take $x \in \emptyset$. By ex falso quodlibet, we have $x \in B$. Therefore $\emptyset \subseteq B$. Thus $B = \emptyset$.\\ \underline{using Thm 5.1}: Assume that $A \cup B = A$ for all $A$. Then, in particular, $\emptyset \cup B = \emptyset$. Then \begin{align*}B &= B \cup \emptyset\tag{5.1.4}\\&= \emptyset \cup B\tag{5.1.2}\\&= \emptyset.\tag{Consequence of assumption}\end{align*} Therefore $B = \emptyset$.\\ \underline{dual proof}: Assume that $A \cap B = A$ for all $A$. Then, in particular, $U \cap B = U$. Then \begin{align*}B &= B \cap U\tag{5.1.4'}\\&= U \cap B\tag{5.1.2'}\\&= U.\tag{Consequence of assumption}\end{align*} Therefore $B = U$.
\item %7%
\underline{using membership}: Assume $A \cup B = U$ and $A \cap B = \emptyset$. Take $x \in B$. Assume $x \in A$. Then $x \in A \cap B = \emptyset$. By contradiction, $x \notin A$ and so $x \in \set{y \in U \mid y \notin A} = U - A = \comp{A}$. Therefore $B \subseteq \comp{A}$. Now take $x \in \comp{A}$. Then $x \notin A$. Assume $x \notin B$. Then $x \notin A \cup B = U$. By contradiction, $x \in B$ and so $\comp{A} \subseteq B$. Therefore $B = \comp{A}$.\\ 
\underline{using Thm 5.1}: Assume $A \cup B = U$ and $A \cap B = \emptyset$. Then we have \begin{align*}B &= B \cap U \tag{5.1.4'}\\&= B \cap (A \cup \comp{A})\tag{5.1.5}\\&= (B \cap A) \cup (B \cap \comp{A})\tag{5.1.3'}\\&= (A \cap B) \cup (B \cap \comp{A})\tag{5.1.2'}\\&=(A \cap B) \cup (\comp{A} \cap B)\tag{5.1.2'}\\&= \emptyset \cup (\comp{A} \cap B) \tag{Assumption}\\&= (A \cap \comp{A}) \cup (\comp{A} \cap B)\tag{5.1.5'}\\&= (\comp{A} \cap A) \cup (\comp{A} \cap B)\tag{5.1.2'}\\&= \comp{A} \cap (A \cup B)\tag{5.1.3'}\\&= \comp{A} \cap U\tag{Assumption}\\&= \comp{A}.\tag{5.1.4'}\end{align*}(self-dual)
\item %8%
\underline{using membership}: Take $x \in \comp{\comp{A}}$. Then $x \in U - \comp{A} = \set{y \in U \mid y \notin \comp{A}} = \set{y \in U \mid y \in A} = A$. Therefore $\comp{\comp{A}} \subseteq A$. Now take $x \in A$. Then $x \notin \comp{A}$ and so $x \in \comp{\comp{A}}$. Then $A \subseteq \comp{\comp{A}}$ and so $\comp{\comp{A}} = A$.\\
\underline{using Thm 5.1}: \begin{align*}\comp{\comp{A}} &= \comp{\comp{A}} \cap U \tag{5.1.4'}\\&= \comp{\comp{A}} \cap (A \cup \comp{A}) \tag{5.1.5}\\&= (\comp{\comp{A}} \cap A) \cup (\comp{\comp{A}} \cap \comp{A}) \tag{5.1.3'}\\&= (\comp{\comp{A}} \cap A) \cup (\comp{A} \cap \comp{\comp{A}}) \tag{5.1.2'}\\&= (\comp{\comp{A}} \cap A) \cup \emptyset \tag{5.1.5'}\\&= \emptyset \cup (\comp{\comp{A}} \cap A)\tag{5.1.2}\\&= (A \cap \comp{A}) \cup (\comp{\comp{A}} \cap A)\tag{5.1.5'}\\&= (A \cap \comp{A}) \cup (A \cap \comp{\comp{A}})\tag{5.1.2'}\\&= A \cap (\comp{A} \cup \comp{\comp{A}})\tag{5.1.3'}\\&= A \cap U\tag{5.1.5}\\&= A.\tag{5.1.4'}\end{align*} (self-dual)
\item %9%
\underline{using membership}: Take $x \in \comp{\emptyset}$. Then $x \in U - \emptyset = \set{y \in U \mid y \notin \emptyset} = \set{y \in U} = U$. Therefore $\comp{\emptyset} \subseteq U$. Now take $x \in U$. Then $x \notin \emptyset$ and so $x \in \comp{\emptyset}$. Therefore $U \subseteq \comp{\emptyset}$. Thus $\comp{\emptyset} = U$.\\
\underline{using Thm 5.1}: \begin{align*}\comp{\emptyset} &= \comp{\emptyset} \cup \emptyset \tag{5.1.4}\\&= \emptyset \cup \comp{\emptyset} \tag{5.1.2}\\&= U.\tag{5.1.5}\end{align*}
\item %10%
\underline{using membership}: Take $x \in A \cup A$. Then $x \in A$ or $x \in A$. Obviously, $x \in A$ and so $A \cup A \subseteq A$. Take $x \in A$. Then $x \in A \cup A$ and so $A \subseteq A \cup A$. Thus $A \cup A = A$.\\
\underline{using Thm 5.1}: \begin{align*}A \cup A &= (A \cup A) \cap U \tag{5.1.4'}\\&= (A \cap A) \cap (A \cup \comp{A})\tag{5.1.5}\\&= A \cup (A \cap \comp{A})\tag{5.1.3}\\&= A \cup \emptyset\tag{5.1.5'}\\&= A.\tag{5.1.4}\end{align*}
\item %11%
\underline{using membership}: Take $x \in A \cup U$. Then $x \in A$ or $x \in U$. Assume $x \notin U$. Then $x \notin A$ since $A \subseteq U$. By contradiction, we must have $x \in U$. Therefore $A \cup U \subseteq U$. Now take $x \in U$. Then $x \in U \cup A = A \cup U$ and therefore $U \subseteq A \cup U$. Thus $A \cup U = U$.\\
\underline{using Thm 5.1}: \begin{align*}A \cup U &= (A \cup U) \cap U\tag{5.1.4'}\\&= U \cap (A \cup U)\tag{5.1.2'}\\&= (A \cup \comp{A}) \cap (A \cup U)\tag{5.1.5}\\&= A \cup (\comp{A} \cap U)\tag{5.1.3}\\&= A \cup \comp{A}\tag{5.1.4'}\\&= U.\tag{5.1.5}\end{align*}
\item %12%
\underline{using membership}: Take $x \in A \cup (A \cap B)$. Then $x \in A$ or $x \in A \cap B$. In the first case, $x \in A$. In the second case, $x \in A$ and $x \in B$. In both cases $x \in A$ and so $A \cup (A \cap B) \subseteq A$. Now take $x \in A$. Then $x \in A \cup (A \cap B)$ and so $A \subseteq A \cup (A \cap B)$. Thus $A \cup (A \cap B) = A$.\\
\underline{using Thm 5.1}: $A = A \cup \emptyset = A \cup ((A \cap B) \cap (\comp{A \cap B})) = A \cup ((A \cap B) \cap (\comp{A} \cup \comp{B})) = (A \cup (A \cap B)) \cap (A \cup (\comp{A} \cup \comp{B})) = (A \cup (A \cap B)) \cap ((A \cup \comp{A}) \cup \comp{B}) = (A \cup (A \cap B)) \cap (U \cup \comp{B}) = (A \cup (A \cap B)) \cap (\comp{B} \cup U) = (A \cup (A \cap B)) \cap U = A \cup (A \cap B)$.
\item %13%
\underline{using membership}: Take $x \in \comp{A \cup B}$. Then $x \notin A \cup B$. Then $x \notin A$ and $x \notin B$ and so $x \in \comp{A}$ and $x \in \comp{B}$ and therefore $x \in \comp{A} \cap \comp{B}$. Thus $\comp{A \cup B} \subseteq \comp{A} \cap \comp{B}$. Now take $x \in \comp{A} \cap \comp{B}$. Then $x \in \comp{A}$ and $x \in \comp{B}$ and so $x \notin A$ and $x \notin B$. Therefore $x \notin A \cup B$ and so $x \in \comp{A \cup B}$. Then we have $\comp{A} \cap \comp{B} \subseteq \comp{A \cup B}$. Thus $\comp{A \cup B} = \comp{A} \cap \comp{B}$.\\
\underline{using Thm 5.1}: \begin{align*}A \cup (A \cap B) &= (A \cap U) \cup (A \cap B)\tag{5.1.4'}\\&= (A \cap (B \cup \comp{B})) \cup (A \cap B)\tag{5.1.5}\\&= ((A \cap B) \cup (A \cap \comp{B})) \cup (A \cap B)\tag{5.1.3'}\\&= (A \cap B) \cup ((A \cap \comp{B}) \cup (A \cap B))\tag{5.1.1}\\&= (A \cap B) \cup ((A \cap B) \cup (A \cap \comp{B}))\tag{5.1.2}\\&= ((A \cap B) \cup (A \cap B)) \cup (A \cap \comp{B})\tag{5.1.1}\\&= (((A \cap B) \cup (A \cap B)) \cap U) \cup (A \cap \comp{B})\tag{5.1.4'}\\&=  (((A \cap B) \cup (A \cap B)) \cap ((A \cap B) \cup (\comp{A \cap B}))) \cup (A \cap \comp{B})\tag{5.1.5}\\&= ((A \cap B) \cup ((A \cap B) \cap (\comp{A \cap B}))) \cup (A \cap \comp{B})\tag{5.1.3}\\&= ((A \cap B) \cup \emptyset) \cup (A \cap \comp{B})\tag{5.1.5'}\\&= (A \cap B) \cup (A \cap \comp{B})\tag{5.1.4}\\&= A \cap (B \cup \comp{B})\tag{5.1.3'}\\&= A \cap U\tag{5.1.5}\\&= A.\tag{5.1.4'}\end{align*}
\end{enumerate}}

\renewcommand{\labelenumii}{(\alph{enumii})}
\item Using only the identities in Theorems 5.1 and 5.2, show that each of the following equations is an identity.
\begin{enumerate}
	\solution{$(A \cap B \cap X) \cup (A \cap B \cap C \cap X \cap Y) \cup (A \cap X \cap \comp{A}) = A \cap B \cap X$.}
	{\begin{align*}&(A \cap B \cap X) \cup (A \cap B \cap C \cap X \cap Y) \cup (A \cap X \cap \comp{A})\\&= [(A \cap B \cap X) \cup (A \cap B \cap C \cap X \cap Y)] \cup (A \cap X \cap \comp{A})\\&= [(A \cap B \cap X) \cup (A \cap B \cap X \cap C \cap Y)] \cup (A \cap X \cap \comp{A})\\&= [(A \cap B \cap X) \cup ((A \cap B \cap X) \cap (C \cap Y))] \cup (A \cap X \cap \comp{A})\\&= (A \cap B \cap X) \cup (A \cap X \cap \comp{A})\\&= (A \cap B \cap X) \cup (A \cap \comp{A} \cap X)\\&= (A \cap B \cap X) \cup (\emptyset \cap X)\\&= (A \cap B \cap X) \cup (X \cap \emptyset)\\&= (A \cap B \cap X) \cup \emptyset\\&= A \cap B \cap X.\end{align*}}
	\solution{$(A \cap B \cap C) \cup (\comp{A} \cap B \cap C) \cup \comp{B} \cup \comp{C} = U$.}
	{\begin{align*}&(A \cap B \cap C) \cup (\comp{A} \cap B \cap C) \cup \comp{B} \cup \comp{C}\\&= (B \cap C \cap A) \cup (B \cap C \cap \comp{A}) \cup \comp{B} \cup \comp{C}\\&= [(B \cap C) \cap A] \cup [(B \cap C) \cap \comp{A}] \cup \comp{B} \cup \comp{C}\\&= [(B \cap C) \cap (A \cup \comp{A})] \cup \comp{B} \cup \comp{C}\\&= [(B \cap C) \cap U] \cup \comp{B} \cup \comp{C}\\&= (B \cap C) \cup \comp{B} \cup \comp{C}\\&= (B \cap C) \cup (\comp{B \cap C})\\&= U.\end{align*}}
	\solution{$(A \cap B \cap C \cap \comp{X}) \cup (\comp{A} \cap C) \cup (\comp{B} \cap C) \cup (C \cap X) = C.$}
	{\begin{align*}&(A \cap B \cap C \cap \comp{X}) \cup (\comp{A} \cap C) \cup (\comp{B} \cap C) \cup (C \cap X)\\&= (C \cap A \cap B \cap \comp{X}) \cup (C \cap \comp{A}) \cup (C \cap \comp{B}) \cup (C \cap X)\\&= (C \cap A \cap B \cap \comp{X}) \cup (C \cap (\comp{A} \cup \comp{B})) \cup (C \cap X)\\&= (C \cap A \cap B \cap \comp{X}) \cup (C \cap [(\comp{A} \cup \comp{B}) \cup X])\\&= (C \cap A \cap B \cap \comp{X}) \cup (C \cap (\comp{A} \cup \comp{B} \cup X))\\&= (C \cap A \cap B \cap \comp{X}) \cup (C \cap (\comp{A \cap B \cup \comp{X}}))\\&= (C \cap (A \cap B \cap \comp{X})) \cup (C \cap (\comp{A \cap B \cup \comp{X}}))\\&= C \cap [(A \cap B \cap \comp{X}) \cup (\comp{A \cap B \cup \comp{X}})]\\&= C \cap U\\&= C.\end{align*}}\pagebreak
	\solution{$[(A \cap B) \cup (A \cap C) \cup (\comp{A} \cap \comp{X} \cap Y)] \cap [\comp{(A \cap \comp{B} \cap C) \cup (\comp{A} \cap \comp{X} \cap \comp{Y}) \cup (\comp{A} \cap B \cap Y)}] = (A \cap B) \cup (\comp{A} \cap \comp{B} \cap \comp{X} \cap Y)$.}
	{In the words of George Costanza: ``Let's get nuts!''
\begin{align*}&[(A \cap B) \cup (A \cap C) \cup (\comp{A} \cap \comp{X} \cap Y)] \cap [\comp{(A \cap \comp{B} \cap C) \cup (\comp{A} \cap \comp{X} \cap \comp{Y}) \cup (\comp{A} \cap B \cap Y)}] \\&= [(A \cap B) \cup (A \cap C) \cup (\comp{A} \cap \comp{X} \cap Y)] \cap [\comp{(A \cap \comp{B} \cap C) \cup (\comp{A} \cap \comp{X} \cap \comp{Y})} \cap (\comp{\comp{A} \cap B \cap Y})]\tag{5.2.13}\\&= [(A \cap B) \cup (A \cap C) \cup (\comp{A} \cap \comp{X} \cap Y)] \cap [(\comp{A \cap \comp{B} \cap C}) \cap (\comp{\comp{A} \cap \comp{X} \cap \comp{Y}}) \cap (\comp{\comp{A} \cap B \cap Y})] \tag{5.2.13}\\&= [(A \cap B) \cup (A \cap C) \cup (\comp{A} \cap \comp{X} \cap Y)] \cap [(\comp{A} \cup \comp{\comp{B}} \cup \comp{C}) \cap (\comp{\comp{A}} \cup \comp{\comp{X}} \cup \comp{\comp{Y}}) \cap (\comp{\comp{A}} \cup \comp{B} \cup \comp{Y})]\tag{5.2.13'}\\&= [(A \cap B) \cup (A \cap C) \cup (\comp{A} \cap \comp{X} \cap Y)] \cap [(\comp{A} \cup B \cup \comp{C}) \cap (A \cup X \cup Y) \cap (A \cup \comp{B} \cup \comp{Y})]\tag{5.2.8}\\&= [(A \cap B) \cup (A \cap C) \cup (\comp{A} \cap \comp{X} \cap Y)] \cap [(\comp{A} \cup B \cup \comp{C}) \cap (A \cup \comp{B} \cup \comp{Y}) \cap (A \cup X \cup Y)]\tag{5.1.2'}\\&= [(A \cap B) \cup (A \cap C) \cup (\comp{A} \cap \comp{X} \cap Y)] \cap [(A \cup \comp{B} \cup \comp{Y}) \cap (\comp{A} \cup B \cup \comp{C}) \cap (A \cup X \cup Y)]\tag{5.1.2'}\\&= [(A \cap B) \cup (A \cap C) \cup (\comp{A} \cap \comp{X} \cap Y)] \cap [(A \cup \comp{B} \cup \comp{Y})  \cap (A \cup X \cup Y) \cap (\comp{A} \cup B \cup \comp{C})]\tag{5.1.2'}\\&= [[[(A \cap B) \cup (A \cap C) \cup (\comp{A} \cap \comp{X} \cap Y)] \cap (A \cup \comp{B} \cup \comp{Y})] \cap (A \cup X \cup Y)] \cap (\comp{A} \cup B \cup \comp{C})\tag{5.1.1'}\end{align*}
I will now break the derivation into three parts, taking care of each of the nested intersections, and taking some reasonable shortcuts along the way. In most places, I've gone above and beyond to make sure that every step appeals to theorem 5.1 or 5.2.\pagebreak
\begin{align*}&[(A \cap B) \cup (A \cap C) \cup (\comp{A} \cap \comp{X} \cap Y)] \cap (A \cup \comp{B} \cup \comp{Y})\\&= (A \cup \comp{B} \cup \comp{Y}) \cap [(A \cap B) \cup (A \cap C) \cup (\comp{A} \cap \comp{X} \cap Y)]\\&= [(A \cup \comp{B} \cup \comp{Y}) \cap (A \cap B)] \cup [(A \cup \comp{B} \cup \comp{Y}) \cap (A \cap C)] \cup [(A \cup \comp{B} \cup \comp{Y}) \cap (\comp{A} \cap \comp{X} \cap Y)]\\&= [((A \cup \comp{B} \cup \comp{Y}) \cap A) \cap B] \cup [((A \cup \comp{B} \cup \comp{Y}) \cap A) \cap C)] \cup [(A \cup \comp{B} \cup \comp{Y}) \cap (\comp{A} \cap \comp{X} \cap Y)]\\&= [(A \cap (A \cup \comp{B} \cup \comp{Y})) \cap B] \cup [(A \cap (A \cup \comp{B} \cup \comp{Y})) \cap C)] \cup [(A \cup \comp{B} \cup \comp{Y}) \cap (\comp{A} \cap \comp{X} \cap Y)]\\&= [(A \cap (A \cup (\comp{B} \cup \comp{Y}))) \cap B] \cup [(A \cap (A \cup (\comp{B} \cup \comp{Y}))) \cap C)] \cup [(A \cup \comp{B} \cup \comp{Y}) \cap (\comp{A} \cap \comp{X} \cap Y)]\\&= (A \cap B) \cup (A \cap C) \cup [(A \cup \comp{B} \cup \comp{Y}) \cap (\comp{A} \cap \comp{X} \cap Y)]\\&= (A \cap B) \cup (A \cap C) \cup [((A \cup \comp{B} \cup \comp{Y}) \cap \comp{A}) \cap \comp{X} \cap Y)]\\&= (A \cap B) \cup (A \cap C) \cup [(\comp{A} \cap (A \cup \comp{B} \cup \comp{Y})) \cap \comp{X} \cap Y)]\\&= (A \cap B) \cup (A \cap C) \cup [((\comp{A} \cap A) \cup (\comp{A} \cap \comp{B}) \cup (\comp{A} \cap \comp{Y})) \cap \comp{X} \cap Y)]\\&= (A \cap B) \cup (A \cap C) \cup [((A \cap \comp{A}) \cup (\comp{A} \cap \comp{B}) \cup (\comp{A} \cap \comp{Y})) \cap \comp{X} \cap Y)]\\&= (A \cap B) \cup (A \cap C) \cup [(\emptyset \cup (\comp{A} \cap \comp{B}) \cup (\comp{A} \cap \comp{Y})) \cap \comp{X} \cap Y)]\\&= (A \cap B) \cup (A \cap C) \cup [((\comp{A} \cap \comp{B}) \cup (\comp{A} \cap \comp{Y}) \cup \emptyset) \cap \comp{X} \cap Y)]\\&= (A \cap B) \cup (A \cap C) \cup [((\comp{A} \cap \comp{B}) \cup (\comp{A} \cap \comp{Y})) \cap \comp{X} \cap Y)]\\&= (A \cap B) \cup (A \cap C) \cup [((\comp{A} \cap \comp{B}) \cup \comp{A}) \cap ((\comp{A} \cap \comp{B}) \cup \comp{Y})) \cap \comp{X} \cap Y)]\\&= (A \cap B) \cup (A \cap C) \cup [(\comp{A} \cup (\comp{A} \cap \comp{B})) \cap (\comp{Y} \cup (\comp{A} \cap \comp{B})) \cap \comp{X} \cap Y)]\\&= (A \cap B) \cup (A \cap C) \cup [(\comp{A} \cap (\comp{Y} \cup (\comp{A} \cap \comp{B})) \cap \comp{X} \cap Y)]\\&= (A \cap B) \cup (A \cap C) \cup [(\comp{A} \cap (\comp{Y} \cup \comp{A}) \cap (\comp{Y} \cup \comp{B})) \cap \comp{X} \cap Y)]\\&= (A \cap B) \cup (A \cap C) \cup [(\comp{A} \cap (\comp{A} \cup \comp{Y}) \cap (\comp{Y} \cup \comp{B})) \cap \comp{X} \cap Y)]\\&= (A \cap B) \cup (A \cap C) \cup [(\comp{A} \cap (\comp{Y} \cup \comp{B})) \cap \comp{X} \cap Y)]\\&= (A \cap B) \cup (A \cap C) \cup [\comp{A} \cap ((\comp{Y} \cup \comp{B}) \cap \comp{X}) \cap Y)]\\&= (A \cap B) \cup (A \cap C) \cup [\comp{A} \cap (\comp{X} \cap (\comp{Y} \cup \comp{B})) \cap Y)]\\&= (A \cap B) \cup (A \cap C) \cup [\comp{A} \cap \comp{X} \cap ((\comp{Y} \cup \comp{B}) \cap Y)]\\&= (A \cap B) \cup (A \cap C) \cup [\comp{A} \cap \comp{X} \cap (Y \cap (\comp{Y} \cup \comp{B}))]\\&= (A \cap B) \cup (A \cap C) \cup [\comp{A} \cap \comp{X} \cap ((Y \cap \comp{Y}) \cup (Y \cap \comp{B}))]\\&= (A \cap B) \cup (A \cap C) \cup [\comp{A} \cap \comp{X} \cap (\emptyset \cup (Y \cap \comp{B}))]\\&= (A \cap B) \cup (A \cap C) \cup [\comp{A} \cap \comp{X} \cap ((Y \cap \comp{B})\cup \emptyset)]\\&= (A \cap B) \cup (A \cap C) \cup [\comp{A} \cap \comp{X} \cap (Y \cap \comp{B})]\\&= (A \cap B) \cup (A \cap C) \cup [\comp{A} \cap \comp{X} \cap (\comp{B} \cap Y)]\\&= (A \cap B) \cup (A \cap C) \cup [\comp{A} \cap (\comp{X} \cap \comp{B}) \cap Y]\\&= (A \cap B) \cup (A \cap C) \cup [\comp{A} \cap (\comp{B} \cap \comp{X}) \cap Y]\\&= (A \cap B) \cup (A \cap C) \cup (\comp{A} \cap \comp{B} \cap \comp{X} \cap Y).\end{align*}
\begin{align*}&[(A \cap B) \cup (A \cap C) \cup (\comp{A} \cap \comp{B} \cap \comp{X} \cap Y)] \cap (A \cup X \cup Y)\\&= (A \cup X \cup Y) \cap [(A \cap B) \cup (A \cap C) \cup (\comp{A} \cap \comp{B} \cap \comp{X} \cap Y)]\\&=[(A \cup X \cup Y) \cap (A \cap B)] \cup [(A \cup X \cup Y) \cap (A \cap C)] \cup [(A \cup X \cup Y) \cap (\comp{A} \cap \comp{B} \cap \comp{X} \cap Y)]\\&= [((A \cup X \cup Y) \cap A) \cap B] \cup [((A \cup X \cup Y) \cap A)\cap C)] \cup [(A \cup X \cup Y) \cap (\comp{A} \cap \comp{B} \cap \comp{X} \cap Y)]\\&= [(A \cap (A \cup X \cup Y)) \cap B] \cup [(A \cap (A \cup X \cup Y))\cap C)] \cup [(A \cup X \cup Y) \cap (\comp{A} \cap \comp{B} \cap \comp{X} \cap Y)]\\&= (A \cap B) \cup (A \cap C) \cup [(A \cup X \cup Y) \cap (\comp{A} \cap \comp{B} \cap \comp{X} \cap Y)]\\&= (A \cap B) \cup (A \cap C) \cup [(A \cup X \cup Y) \cap (Y \cap \comp{A} \cap \comp{B} \cap \comp{X})]\\&= (A \cap B) \cup (A \cap C) \cup [((A \cup X \cup Y) \cap Y) \cap \comp{A} \cap \comp{B} \cap \comp{X}]\\&= (A \cap B) \cup (A \cap C) \cup [(Y \cap (A \cup X \cup Y)) \cap \comp{A} \cap \comp{B} \cap \comp{X}]\\&= (A \cap B) \cup (A \cap C) \cup [(Y \cap (Y \cup A \cup X)) \cap \comp{A} \cap \comp{B} \cap \comp{X}]\\&= (A \cap B) \cup (A \cap C) \cup (Y \cap \comp{A} \cap \comp{B} \cap \comp{X})\\&= (A \cap B) \cup (A \cap C) \cup (\comp{A} \cap \comp{B} \cap \comp{X} \cap Y).\end{align*}
\begin{align*}&[(A \cap B) \cup (A \cap C) \cup (\comp{A} \cap \comp{B} \cap \comp{X} \cap Y)] \cap (\comp{A} \cup B \cup \comp{C})\\&= [(A \cap B) \cap (\comp{A} \cup B \cup \comp{C})] \cup [(A \cap C) \cap (\comp{A} \cup B \cup \comp{C})] \cup [(\comp{A} \cap \comp{B} \cap \comp{X} \cap Y) \cap (\comp{A} \cup B \cup \comp{C})]\\&= [A \cap (B \cap (\comp{A} \cup B \cup \comp{C}))] \cup [(A \cap C) \cap (\comp{A} \cup B \cup \comp{C})] \cup [(\comp{A} \cap \comp{B} \cap \comp{X} \cap Y) \cap (\comp{A} \cup B \cup \comp{C})]\\&= [A \cap (B \cap (B \cup \comp{A} \cup \comp{C}))] \cup [(A \cap C) \cap (\comp{A} \cup B \cup \comp{C})] \cup [(\comp{A} \cap \comp{B} \cap \comp{X} \cap Y) \cap (\comp{A} \cup B \cup \comp{C})]\\&= (A \cap B) \cup [(A \cap C) \cap (\comp{A} \cup B \cup \comp{C})] \cup [(\comp{A} \cap \comp{B} \cap \comp{X} \cap Y) \cap (\comp{A} \cup B \cup \comp{C})]\\&= (A \cap B) \cup [(A \cap C) \cap (\comp{A} \cup \comp{C} \cup B)] \cup [(\comp{A} \cap \comp{B} \cap \comp{X} \cap Y) \cap (\comp{A} \cup B \cup \comp{C})]\\&= (A \cap B) \cup [(A \cap C) \cap (\comp{A \cap C} \cup B)] \cup [(\comp{A} \cap \comp{B} \cap \comp{X} \cap Y) \cap (\comp{A} \cup B \cup \comp{C})]\\&= (A \cap B) \cup [((A \cap C) \cap (\comp{A \cap C})) \cup ((A \cap C) \cap B)] \cup [(\comp{A} \cap \comp{B} \cap \comp{X} \cap Y) \cap (\comp{A} \cup B \cup \comp{C})]\\&= (A \cap B) \cup [\emptyset \cup ((A \cap C) \cap B)] \cup [(\comp{A} \cap \comp{B} \cap \comp{X} \cap Y) \cap (\comp{A} \cup B \cup \comp{C})]\\&= (A \cap B) \cup [((A \cap C) \cap B) \cup \emptyset] \cup [(\comp{A} \cap \comp{B} \cap \comp{X} \cap Y) \cap (\comp{A} \cup B \cup \comp{C})]\\&= (A \cap B) \cup ((A \cap C) \cap B) \cup [(\comp{A} \cap \comp{B} \cap \comp{X} \cap Y) \cap (\comp{A} \cup B \cup \comp{C})]\\&= (A \cap B) \cup (A \cap (C \cap B)) \cup [(\comp{A} \cap \comp{B} \cap \comp{X} \cap Y) \cap (\comp{A} \cup B \cup \comp{C})]\\&= (A \cap B) \cup (A \cap (B \cap C)) \cup [(\comp{A} \cap \comp{B} \cap \comp{X} \cap Y) \cap (\comp{A} \cup B \cup \comp{C})]\\&= (A \cap B) \cup ((A \cap B) \cap C) \cup [(\comp{A} \cap \comp{B} \cap \comp{X} \cap Y) \cap (\comp{A} \cup B \cup \comp{C})]\\&= (A \cap B) \cup [(\comp{A} \cap \comp{B} \cap \comp{X} \cap Y) \cap (\comp{A} \cup B \cup \comp{C})]\\&= (A \cap B) \cup [((\comp{A} \cap \comp{B} \cap \comp{X} \cap Y) \cap \comp{A}) \cup ((\comp{A} \cap \comp{B} \cap \comp{X} \cap Y) \cap B) \cup ((\comp{A} \cap \comp{B} \cap \comp{X} \cap Y) \cap \comp{C})]\\&= (A \cap B) \cup [(\comp{A} \cap (\comp{A} \cap \comp{B} \cap \comp{X} \cap Y)) \cup (B \cap (\comp{A} \cap \comp{B} \cap \comp{X} \cap Y)) \cup (\comp{C} \cap (\comp{A} \cap \comp{B} \cap \comp{X} \cap Y))]\\&= (A \cap B) \cup [((\comp{A} \cap \comp{A}) \cap \comp{B} \cap \comp{X} \cap Y) \cup (B \cap (\comp{A} \cap \comp{B} \cap \comp{X} \cap Y)) \cup (\comp{C} \cap (\comp{A} \cap \comp{B} \cap \comp{X} \cap Y))]\\&= (A \cap B) \cup [(\comp{A} \cap \comp{B} \cap \comp{X} \cap Y) \cup (B \cap (\comp{A} \cap \comp{B} \cap \comp{X} \cap Y)) \cup (\comp{C} \cap (\comp{A} \cap \comp{B} \cap \comp{X} \cap Y))]\\&= (A \cap B) \cup [(\comp{A} \cap \comp{B} \cap \comp{X} \cap Y) \cup (B \cap (\comp{B} \cap \comp{A} \cap \comp{X} \cap Y)) \cup (\comp{C} \cap (\comp{A} \cap \comp{B} \cap \comp{X} \cap Y))]\\&= (A \cap B) \cup [(\comp{A} \cap \comp{B} \cap \comp{X} \cap Y) \cup ((B \cap \comp{B}) \cap \comp{A} \cap \comp{X} \cap Y) \cup (\comp{C} \cap (\comp{A} \cap \comp{B} \cap \comp{X} \cap Y))]\\&= (A \cap B) \cup [(\comp{A} \cap \comp{B} \cap \comp{X} \cap Y) \cup (\emptyset \cap \comp{A} \cap \comp{X} \cap Y) \cup (\comp{C} \cap (\comp{A} \cap \comp{B} \cap \comp{X} \cap Y))]\\&= (A \cap B) \cup [(\comp{A} \cap \comp{B} \cap \comp{X} \cap Y) \cup (\comp{A} \cap \comp{X} \cap Y \cap \emptyset) \cup (\comp{C} \cap (\comp{A} \cap \comp{B} \cap \comp{X} \cap Y))]\\&= (A \cap B) \cup [(\comp{A} \cap \comp{B} \cap \comp{X} \cap Y) \cup \emptyset \cup (\comp{C} \cap (\comp{A} \cap \comp{B} \cap \comp{X} \cap Y))]\\&= (A \cap B) \cup [(\comp{A} \cap \comp{B} \cap \comp{X} \cap Y) \cup (\comp{C} \cap (\comp{A} \cap \comp{B} \cap \comp{X} \cap Y)) \cup \emptyset]\\&= (A \cap B) \cup [(\comp{A} \cap \comp{B} \cap \comp{X} \cap Y) \cup (\comp{C} \cap (\comp{A} \cap \comp{B} \cap \comp{X} \cap Y))]\\&= (A \cap B) \cup [(\comp{A} \cap \comp{B} \cap \comp{X} \cap Y) \cup ((\comp{A} \cap \comp{B} \cap \comp{X} \cap Y) \cap \comp{C})]\\&= (A \cap B) \cup (\comp{A} \cap \comp{B} \cap \comp{X} \cap Y).\end{align*}}
\end{enumerate}

\solution{Rework Exercise 4.9(b), using solely the algebra of sets developed in this section.}
{Exercise 4.9(b) states: Using a Venn diagram, show that the symmetric difference is a commutative and associative operation. In the language of the algebra of sets, this translates to: For all sets $A,B$ and $C$ show that $A + B = B + A$ and that $A + (B + C) = (A + B) + C$.
\begin{itemize}
\item \thm{Theorem}: $A + B = B + A$.\\
\thm{Proof}: \begin{align*}A + B &= (A - B) \cup (B - A)\tag{Definition of symmetric difference}\\&=
(B - A) \cup (A - B)\tag{5.1.2}\\&= 
B + A.\tag{Definition of symmetric difference}\end{align*}
\item \thm{Theorem}: $A + (B + C) = (A + B) + C$.\\
In order to prove the theorem, the following few lemmas will be useful:\\
\thm{Lemma 1}: $(A - B) - C = A - (B \cup C)$.\\
\thm{Proof}: \begin{align*}(A - B) - C &= (A \cap \comp{B}) - C \tag{Definition of relative complement}\\&= (A \cap \comp{B}) \cap \comp{C}\tag{Definition of relative complement}\\&= A \cap (\comp{B} \cap \comp{C}) \tag{5.1.1'}\\&= A \cap (\comp{B \cup C})\tag{5.1.13}\\&= A - (B \cup C)\tag{Definition of relative complement}\end{align*} 

\thm{Lemma 2}: $(A - C) \cup (B - C) = (A \cup B) - C$.\\
\thm{Proof}: \begin{align*}(A - C) \cup (B - C) &= (A \cap \comp{C}) \cup (B \cap \comp{C})\tag{Definition of relative complement}\\&= (\comp{C} \cap A) \cup (\comp{C} \cap B)\tag{5.1.2'}\\&= \comp{C} \cap (A \cup B)\tag{5.1.3'}\\&= (A \cup B) \cap \comp{C}\tag{5.1.2'}\\&= (A \cup B) - C \tag{Definition of relative complement}\end{align*}

\thm{Lemma 3}: $(A \cup \comp{B}) \cap (\comp{A} \cup B) = (A \cap B) \cup (\comp{A} \cap \comp{B})$.\\
\thm{Proof}: \begin{align*} (A \cup \comp{B}) \cap (\comp{A} \cup B) &= [(A \cup \comp{B}) \cap \comp{A}] \cup [(A \cup \comp{B}) \cap B] \tag{5.1.3'}\\&= [\comp{A} \cap (A \cup \comp{B})] \cup [B \cap (A \cup \comp{B})]\tag{5.1.2'}\\&= [(\comp{A} \cap A) \cup (\comp{A} \cap \comp{B})] \cup [(B \cap A) \cup (B \cap \comp{B})]\tag{5.1.3'}\\&= [(A \cap \comp{A}) \cup (\comp{A} \cap \comp{B})] \cup [(A \cap B) \cup (B \cap \comp{B})]\tag{5.1.2'}\\&= [(\comp{A} \cap \comp{B}) \cup (A \cap \comp{A})] \cup [(A \cap B) \cup (B \cap \comp{B})]\tag{5.1.2}\\&= [(A \cap B) \cup (B \cap \comp{B})] \cup [(\comp{A} \cap \comp{B}) \cup (A \cap \comp{A})]\tag{5.1.2}\\&= [(A \cap B) \cup \emptyset] \cup [(\comp{A} \cap \comp{B}) \cup \emptyset]\tag{5.1.5'}\\&= (A \cap B) \cup (\comp{A} \cap \comp{B})\tag{5.1.4}\end{align*}

\thm{Proof of theorem}:
\begin{align*} A + (B + C) &= A + [(B - C) \cup (C - B)]\tag{Definition of symmetric difference}\\&= 
(A - [(B - C) \cup (C - B)]) \cup ([(B - C) \cup (C - B)] - A)\tag{Definition of symmetric difference}\\&= 
([A - (B - C)] - [C - B]) \cup ([(B - C) \cup (C - B)] - A)\tag{Lemma 1}\\&= 
([A - (B - C)] - [C - B]) \cup ([(B - C) - A] \cup [(C - B) - A])\tag{Lemma 2}\\&= 
([A - (B \cap \comp{C})] - [C \cap \comp{B}]) \cup ([(B \cap \comp{C}) - A] \cup [(C \cap \comp{B}) - A])\tag{Definition of relative complement}\\&= 
([A - (B \cap \comp{C})] - [C \cap \comp{B}]) \cup ([(B \cap \comp{C}) \cap \comp{A}] \cup [(C \cap \comp{B}) \cap \comp{A}])\tag{Definition of relative complement}\\&= 
([A - (B \cap \comp{C})] - [C \cap \comp{B}]) \cup ([\comp{A} \cap (B \cap \comp{C})] \cup [\comp{A} \cap (C \cap \comp{B})])\tag{5.1.2'}\\&= 
([A - (B \cap \comp{C})] - [C \cap \comp{B}]) \cup ([\comp{A} \cap (B \cap \comp{C})] \cup [\comp{A} \cap (\comp{B} \cap C)])\tag{5.1.2'}\\&= 
([A - (B \cap \comp{C})] - [C \cap \comp{B}]) \cup (\comp{A} \cap B \cap \comp{C}) \cup (\comp{A} \cap \comp{B} \cap C)\tag{5.1.1'}\\&= 
(A - [(B \cap \comp{C}) \cup (C \cap \comp{B})]) \cup (\comp{A} \cap B \cap \comp{C}) \cup (\comp{A} \cap \comp{B} \cap C)\tag{Lemma 1}\\&= 
(A \cap [\comp{(B \cap \comp{C}) \cup (C \cap \comp{B})}]) \cup (\comp{A} \cap B \cap \comp{C}) \cup (\comp{A} \cap \comp{B} \cap C)\tag{Definition of relative complement}\\&= 
(A \cap [(\comp{B \cap \comp{C}}) \cap (\comp{C \cap \comp{B}})]) \cup (\comp{A} \cap B \cap \comp{C}) \cup (\comp{A} \cap \comp{B} \cap C)\tag{5.2.13}\\&= 
(A \cap [(\comp{B} \cup \comp{\comp{C}}) \cap (\comp{C} \cup \comp{\comp{B}})]) \cup (\comp{A} \cap B \cap \comp{C}) \cup (\comp{A} \cap \comp{B} \cap C)\tag{5.2.13'}\\&= 
(A \cap [(\comp{B} \cup C) \cap (\comp{C} \cup B)]) \cup (\comp{A} \cap B \cap \comp{C}) \cup (\comp{A} \cap \comp{B} \cap C)\tag{5.2.8}\\&= 
(A \cap [(\comp{B} \cup C) \cap (B \cup \comp{C})]) \cup (\comp{A} \cap B \cap \comp{C}) \cup (\comp{A} \cap \comp{B} \cap C)\tag{5.1.2}\\&= 
(A \cap [(B \cap C) \cup (\comp{B} \cap \comp{C})]) \cup (\comp{A} \cap B \cap \comp{C}) \cup (\comp{A} \cap \comp{B} \cap C)\tag{Lemma 3}\\&= 
[(A \cap (B \cap C)) \cup (A \cap (\comp{B} \cap \comp{C}))] \cup (\comp{A} \cap B \cap \comp{C}) \cup (\comp{A} \cap \comp{B} \cap C)\tag{5.1.3'}
\end{align*} 
 
\begin{align*}&=[(A \cap B \cap C) \cup (A \cap \comp{B} \cap \comp{C})] \cup (\comp{A} \cap B \cap \comp{C}) \cup (\comp{A} \cap \comp{B} \cap C)\tag{5.1.1'}\\&= 
(A \cap B \cap C) \cup (A \cap \comp{B} \cap \comp{C}) \cup (\comp{A} \cap B \cap \comp{C}) \cup (\comp{A} \cap \comp{B} \cap C)\tag{5.1.1}\\&= 
(A \cap \comp{B} \cap \comp{C}) \cup (A \cap B \cap C) \cup (\comp{A} \cap B \cap \comp{C}) \cup (\comp{A} \cap \comp{B} \cap C)\tag{5.1.2}\\&=
(A \cap \comp{B} \cap \comp{C}) \cup (\comp{A} \cap B \cap \comp{C}) \cup (A \cap B \cap C) \cup (\comp{A} \cap \comp{B} \cap C)\\&= 
[(A \cap \comp{B} \cap \comp{C}) \cup (\comp{A} \cap B \cap \comp{C})] \cup [(A \cap B \cap C) \cup (\comp{A} \cap \comp{B} \cap C)]\tag{5.1.1}\\&= 
[(A \cap \comp{B} \cap \comp{C}) \cup (\comp{A} \cap B \cap \comp{C})] \cup [(C \cap A \cap B) \cup (C \cap \comp{A} \cap \comp{B})]\tag{5.1.2'}\\&= 
[(A \cap \comp{B} \cap \comp{C}) \cup (\comp{A} \cap B \cap \comp{C})] \cup [C \cap [(A \cap B) \cup (\comp{A} \cap \comp{B})]]\tag{5.1.3'}\\&= 
[(A \cap \comp{B} \cap \comp{C}) \cup (\comp{A} \cap B \cap \comp{C})] \cup [C \cap [(A \cup \comp{B}) \cap (\comp{A} \cup B)]]\tag{Lemma 3}\\&= 
[(A \cap \comp{B} \cap \comp{C}) \cup (\comp{A} \cap B \cap \comp{C})] \cup [[C \cap (A \cup \comp{B})] \cap (\comp{A} \cup B)]\tag{5.1.1'}\\&= 
[(A \cap \comp{B} \cap \comp{C}) \cup (\comp{A} \cap B \cap \comp{C})] \cup [[C \cap (\comp{A} \cup B)] \cap (A \cup \comp{B})]\tag{5.1.2'}\\&=
[(A \cap \comp{B} \cap \comp{C}) \cup (\comp{A} \cap B \cap \comp{C})] \cup [[C \cap (\comp{A} \cup B)] \cap (\comp{\comp{A}} \cup \comp{B})]\tag{5.2.8}\\&=
[(A \cap \comp{B} \cap \comp{C}) \cup (\comp{A} \cap B \cap \comp{C})] \cup [[C \cap (\comp{A} \cup B)] \cap (\comp{\comp{A} \cap B})]\tag{5.2.13'}\\&=
[(A \cap \comp{B} \cap \comp{C}) \cup (\comp{A} \cap B \cap \comp{C})] \cup [[C \cap (\comp{A} \cup B)] - (\comp{A} \cap B)]\tag{Definition of relative complement}\\&=
[(A \cap \comp{B} \cap \comp{C}) \cup (\comp{A} \cap B \cap \comp{C})] \cup [[C \cap (\comp{A} \cup \comp{\comp{B}})] - (\comp{A} \cap B)]\tag{5.2.8}\\&=
[(A \cap \comp{B} \cap \comp{C}) \cup (\comp{A} \cap B \cap \comp{C})] \cup [[C \cap (\comp{A \cap \comp{B}})] - (\comp{A} \cap B)]\tag{5.2.13'}\\&=
[(A \cap \comp{B} \cap \comp{C}) \cup (\comp{A} \cap B \cap \comp{C})] \cup [[C - (A \cap \comp{B})] - (\comp{A} \cap B)]\tag{Definition of relative complement}\\&=
[(A \cap \comp{B} \cap \comp{C}) \cup (\comp{A} \cap B \cap \comp{C})] \cup [[C - (A \cap \comp{B})] - (B \cap \comp{A})]\tag{5.1.2'}\\&=
[(A \cap \comp{B} \cap \comp{C}) \cup (\comp{A} \cap B \cap \comp{C})] \cup [[C - (A - B)] - (B - A)]\tag{Definition of relative complement}\\&=
[(A \cap \comp{B} \cap \comp{C}) \cup (\comp{A} \cap B \cap \comp{C})] \cup [C - [(A - B) \cup (B - A)]]\tag{Lemma 1}\\&=
[((A \cap \comp{B}) \cap \comp{C}) \cup ((\comp{A} \cap B) \cap \comp{C})] \cup [C - [(A - B) \cup (B - A)]]\tag{5.1.1'}\\&=
[((A \cap \comp{B}) - C) \cup ((\comp{A} \cap B) - C)] \cup [C - [(A - B) \cup (B - A)]]\tag{Definition of relative complement}\\&=
[((A \cap \comp{B}) - C) \cup ((B \cap \comp{A}) - C)] \cup [C - [(A - B) \cup (B - A)]]\tag{5.1.2'}\\&=
[((A - B) - C) \cup ((B - A) - C)] \cup [C - [(A - B) \cup (B - A)]]\tag{Definition of relative complement}\\&=
[[(A - B) \cup (B - A)] - C] \cup [C - [(A - B) \cup (B - A)]]\tag{Lemma 2}\\&=
[(A + B) - C] \cup [C - (A + B)]\tag{Definition of symmetric difference}\\&=
(A + B) + C\tag{Definition of symmetric difference}
\end{align*}
\end{itemize}}
\pagebreak
\solution{Let $A_1, A_2, \dots, A_n$ be sets, and define $S_k$ to be $A_1 \cup A_2 \cup \dots \cup A_k$ for $k = 1, 2, \dots, n$. Show that $$\mathcal{A} = \set{A_1, A_2 - S_1, A_3 - S_2, \dots, A_n - S_n}$$ is a disjoint collection of sets and that $$S_n = \set{A_1 \cup (A_2 - S_1) \cup \dots \cup (A_n - S_{n-1})}.$$ When is $\mathcal{A}$ a partition of $S_n$?}
{First we will extend the definition of $S_k$ to include the case of\linebreak $k=0$: Define $S_0 = \emptyset$. Then we may write $A_1 = A_1 - S_0$.\\ Now take any two distinct elements $B_1 = A_i - S_{i-1}$ and $B_2 = A_j - S_{j-1}$ of $\mathcal{A}$ where, without loss of generality, $1 \leq i < j \leq n$. Then $S_{j-1} = A_1 \cup \dots \cup A_i \cup \dots \cup A_{j-1}$ and so we have $$B_2 = A_j - S_{j-1} = A_j \cap (\comp{A_1 \cup \dots \cup A_i \cup \dots \cup A_{j-1}}) = A_j \cap \comp{A_1} \cap \dots \cap \comp{A_i} \cap \dots \cap \comp{A_{j-1}}.$$ Therefore $B_1 \cap B_2 = (A_i - S_{i-1}) \cap (A_j - S_{j-1}) = (A_i \cap \comp{S_{i-1}}) \cap (A_j \cap \comp{A_1} \cap \dots \cap \comp{A_i} \cap \dots \cap \comp{A_{j-1}}) = A_i \cap \comp{A_i} \cap (\comp{S_{i-1}} \cap \comp{A_1} \cap \dots \cap \comp{A_{i-1}} \cap \comp{A_{i+1}} \cap \dots \cap \comp{A_{j-1}}) = \emptyset \cap (\comp{S_{i-1}} \cap \comp{A_1} \cap \dots \cap \comp{A_{i-1}} \cap \comp{A_{i+1}} \cap \dots \cap \comp{A_{j-1}}) = \emptyset$ and thus $\mathcal{A}$ is a disjoint collection of sets.\\

To prove $S_n = \set{A_1 \cup (A_2 - S_1) \cup \dots \cup (A_n - S_{n-1})}$, first we show that \begin{align*}A_1 \cup (A_2 - S_1) &= A_1 \cup (A_2 \cap \comp{S_1})\\&= A_1 \cup (A_2 \cap \comp{A_1})\\&= (A_1 \cup A_2) \cap (A_1 \cap \comp{A_1})\\&= (A_1 \cup A_2) \cap \emptyset\\&= A_1 \cup A_2.\end{align*} Now suppose $A_1 \cup (A_2 - S_1) \cup \dots \cup (A_k - S_{k-1}) = A_1 \cup A_2 \cup \dots \cup A_k$ for some $k>2$. Then \begin{align*}A_1 \cup (A_2 - S_1) &\cup \dots \cup (A_k - S_{k-1}) \cup (A_{k+1} - S_k)\\&= A_1 \cup A_2 \cup \dots \cup A_k \cup (A_{k+1} - S_k)\\&= A_1 \cup A_2 \cup \dots \cup A_k \cup (A_{k+1} \cap \comp{S_k})\\&= A_1 \cup A_2 \cup \dots \cup A_k \cup (A_{k+1} \cap \comp{A_1 \cup A_2 \cup \dots \cup A_k})\\&= (A_1 \cup A_2 \cup \dots \cup A_k \cup A_{k+1}) \cap (A_1 \cup A_2 \cup \dots \cup A_k \cup \comp{A_1 \cup A_2 \cup \dots \cup A_k})\\&= (A_1 \cup A_2 \cup \dots \cup A_k \cup A_{k+1}) \cap U\\&= A_1 \cup A_2 \cup \dots \cup A_k \cup A_{k+1}.\end{align*} Therefore for any given $n$, $S_n = \set{A_1 \cup (A_2 - S_1) \cup \dots \cup (A_n - S_{n-1})}$.\\

$\mathcal{A}$ is a partition of $S_n$ as long as all of $A_i$ for $1\leq i \leq n$ are nonempty. (This is all we need since we already have that $\mathcal{A}$ is a disjoint collection of sets.) We have that $A_i$ is empty whenever $A_i \subseteq A_j$ for some $j < i$ since $A_i - S_{i-1} = A_i \cap (\comp{A_1} \cap \dots \cap \comp{A_j} \cap \dots \cap \comp{A_{i-1}}) = A_i \cap \comp{A_j} \cap \dots = \emptyset \cap \dots = \emptyset.$}

\solution{Prove that for arbitrary sets $A_1, A_2, \dots, A_n$ ($n \geq 2$), \begin{align*}A_1 \cup A_2 \cup \dots \cup A_n = (A_1 - A_2) \cup & (A_2 - A_3) \cup \dots \cup (A_{n-1} - A_n)\\&\cup (A_n - A_1) \cup (A_1 \cap A_2 \cap \dots \cap A_n).\end{align*}}
{Take $x \in A_1 \cup A_2 \cup \dots \cup A_n$. Then $x \in A_i$ for some $1 \leq i \leq n$. If $x \notin A_{i+1}$ then we have $x \in (A_i - A_{i+1})$ (if $i=n$ substitute 1 for $i+1$ - i.e.\ arithmetic on the set indices is modulo $n$). Otherwise, if $x \notin A_{i+2}$, then $x \in (A_{i+1} - A_{i+2})$. If not, continue in this manner, checking whether $x \in A_{i+3},A_{i+4}$, etc. If for all $1 \leq j \leq n$ we have $x \in A_j$ but $x \notin (A_j - A_{j+1})$ then we must have $x \in A_1 \cap A_2 \cap \dots \cap A_n$. In any case, $x \in (A_1 - A_2) \cup (A_2 - A_3) \cup \dots \cup (A_{n-1} - A_n) \cup (A_n - A_1) \cup (A_1 \cap A_2 \cap \dots \cap A_n)$ and so $A_1 \cup A_2 \cup \dots \cup A_n \subseteq (A_1 - A_2) \cup (A_2 - A_3) \cup \dots \cup (A_{n-1} - A_n) \cup (A_n - A_1) \cup (A_1 \cap A_2 \cap \dots \cap A_n)$. Now take $x \in (A_1 - A_2) \cup (A_2 - A_3) \cup \dots \cup (A_{n-1} - A_n) \cup (A_n - A_1) \cup (A_1 \cap A_2 \cap \dots \cap A_n)$. Then $x \in A_i - A_{i+1}$ for some $1 \leq i \leq n$ (again substituting 1 for $i+1$ when $i=n$) or $x \in A_1 \cap A_2 \cap \dots \cap A_n$. If $x \in A_1 \cap A_2 \cap \dots \cap A_n$ then clearly $x \in A_1 \cup A_2 \cup \dots \cup A_n$. Now if $x \in A_i - A_{i+1}$ for some $i$ then clearly $x \in A_i$ and so $x \in A_1 \cup A_2 \cup \dots \cup A_n$. Therefore $(A_1 - A_2) \cup (A_2 - A_3) \cup \dots \cup (A_{n-1} - A_n) \cup (A_n - A_1) \cup (A_1 \cap A_2 \cap \dots \cap A_n) \subseteq A_1 \cup A_2 \cup \dots \cup A_n$ and thus $A_1 \cup A_2 \cup \dots \cup A_n = (A_1 - A_2) \cup (A_2 - A_3) \cup \dots \cup (A_{n-1} - A_n) \cup (A_n - A_1) \cup (A_1 \cap A_2 \cap \dots \cap A_n)$.}

\item Referring to Example 5.2, prove the following.
\begin{enumerate}
\solution{For all sets $A$ and $B$, $A = B$ iff $A + B = \emptyset$.}
{Let $A = B$. Then $A + B = (A - B) \cup (B - A) = (A - A) \cup (B - B) = \emptyset \cup \emptyset = \emptyset$. Now let $A + B = \emptyset$. Then $(A - B) \cup (B - A) = \emptyset$. Thus $A - B = \emptyset$ and so $A = B$.}

\solution{An equation in $X$ with righthand member $\emptyset$ can be reduced to one of the form $(A \cap X) \cup (B \cap \comp{X}) = \emptyset$. (Suggestion: Sketch a proof along these lines. First, apply the DeMorgan laws until only complements of individual sets appear. Then expand the resulting lefthand side by the distributive law 3 so as to transform it into the union of several terms $T_n$ each of which is an intersection of several individual sets. Next, if in any $T_i$, neither $X$ nor $\comp{X}$ appears, replace $T_i$ by $T_i \cap (X \cup \comp{X})$ and expand. Finally, group together the terms containing $X$ and those containing $\comp{X}$ and apply distributive law 3'.)}
{...}

\solution{For all sets $A$ and $B$, $A = B = \emptyset$ iff $A \cup B = \emptyset$.}
{Let $A = B = \emptyset$. Then $A \cup B = \emptyset \cup \emptyset = \emptyset$. Now let $A \cup B = \emptyset$. Then $A = \emptyset$ and $B = \emptyset$.}

\solution{The equation $(A \cap X) \cup (B \cap \comp{X}) = \emptyset$ has a solution iff $B \subseteq \comp{A}$, and then any $X$ such that $B \subseteq X \subseteq \comp{A}$ is a solution.}
{($\then$) Suppose there exists a set $X$ for which $(A \cap X) \cup (B \cap \comp{X}) = \emptyset$. Now consider some $x \in B$ and assume $x \in A$. Then if $x \in X$ we have $x \in A \cap X \subseteq (A \cap X) \cup (B \cap \comp{X}) = \emptyset$, a contradiction. Otherwise, $x \in \comp{X}$ and so $x \in B \cap \comp{X} \subseteq (A \cap X) \cup (B \cap \comp{X}) = \emptyset$, another contradiction. Therefore, $x \notin A$ and so $B \subseteq \comp{A}$.\\ ($\Leftarrow$) Now assuming $B \subseteq \comp{A}$ consider a set $X$ such that $B \subseteq X \subseteq \comp{A}$. Then for any $x \in X$, $x \in \comp{A} \then x \notin A \then X \cap A = \emptyset$. Also, for any $x \in B$, $x \in X$. Therefore $x \notin \comp{X}$ and so $B \cap \comp{X} = \emptyset$. Therefore $(A \cap X) \cup (B \cap \comp{X}) = \emptyset$.}

\solution{An alternative form for solutions of the equation in part (d) is $X = (B \cup T) \cap \comp{A}$, where $T$ is an arbitrary set.}
{Let $X = (B \cup T) \cap \comp{A}$ for an arbitrary set $T$. Then we have $\comp{X} = (\comp{B} \cap \comp{T}) \cup A$, $A \cap X = A \cap [(B \cup T) \cap \comp{A}] = A \cap \comp{A} \cap (B \cup T) = \emptyset \cap (B \cup T) = \emptyset$, and $B \cap \comp{X} = B \cap [(\comp{B} \cap \comp{T}) \cup A] = (B \cap \comp{B} \cap \comp{T}) \cup (B \cap A) = \emptyset \cap \emptyset = \emptyset$. Therefore $(A \cap X) \cup (B \cap \comp{X}) = \emptyset$ and so $X = (B \cup T) \cap \comp{A}$ is alternative form for solutions of the equation in part (d).}
\end{enumerate}

\item Show that for arbitrary sets $A, B, C, D$, and $X$,
\begin{enumerate}
\solution{$\comp{[(A \cap X) \cup (B \cap \comp{X})]} = (\comp{A} \cap X) \cup (\comp{B} \cap \comp{X})$}
{$\comp{[(A \cap X) \cup (B \cap \comp{X})]} = \comp{A \cap X} \cap \comp{B \cap \comp{X}} = (\comp{A} \cup \comp{X}) \cap (\comp{B} \cup X))$.\\Since $X$ is an arbitrary set we may reassign $X \leftarrow \comp{X}$.\\Then we have $\comp{[(A \cap X) \cup (B \cap \comp{X})]} = (\comp{A} \cup X) \cap (\comp{B} \cup \comp{X})$.}

\solution{$[(A \cap X) \cup (B \cap \comp{X})] \cup [(C \cap X) \cup (D \cap \comp{X})] = [(A \cup C) \cap X] \cup [(B \cup D) \cap \comp{X}]$}
{\begin{align*}[(A \cap X) \cup (B \cap \comp{X})] \cup [(C \cap X) \cup (D \cap \comp{X})] &= (A \cap X) \cup (B \cap \comp{X}) \cup (C \cap X) \cup (D \cap \comp{X})\\&= (A \cap X) \cup (C \cap X) \cup (B \cap \comp{X}) \cup (D \cap \comp{X})\\&= [(A \cup C) \cap X] \cup [(B \cup D) \cap \comp{X}]\end{align*}}

\solution{$[(A \cap X) \cup (B \cap \comp{X})] \cap [(C \cap X) \cup (D \cap \comp{X})] = [(A \cap C) \cap X] \cup [(B \cap D) \cap \comp{X}]$}
{\begin{align*}&[(A \cap X) \cup (B \cap \comp{X})] \cap [(C \cap X) \cup (D \cap \comp{X})]\\ &= \left[[(A \cap X) \cup (B \cap \comp{X})] \cap (C \cap X)\right] \cup \left[[(A \cap X) \cup (B \cap \comp{X})] \cap (D \cap \comp{X})\right]\\&= \left[[(A \cap X) \cap (C \cap X)] \cup [(B \cap \comp{X}) \cap (C \cap X)]\right]\\&\hspace{0.7cm}\cup \left[[(A \cap X) \cap (D \cap \comp{X})] \cup [(B \cap \comp{X}) \cap (D \cap \comp{X})]\right] \\&= (A \cap C \cap X) \cup (B \cap C \cap X \cap \comp{X}) \cup (A \cap D \cap X \cap \comp{X}) \cup (B \cap D \cap \comp{X})\\&= (A \cap C \cap X) \cup \emptyset \cup \emptyset \cup (B \cap D \cap \comp{X}) \\&= (A \cap C \cap X) \cup (B \cap D \cap \comp{X}) \\&= [(A \cap C) \cap X] \cup [(B \cap D) \cap \comp{X}] \end{align*}}
\end{enumerate}

\solution{Using the results in Exercises 5.7 and 5.8, prove that the equation $$(A \cap X) \cup (B \cap \comp{X}) = (C \cap X) \cup (D \cup \comp{X})$$ has a solution iff $B + D \subseteq \comp{A + C}$. In this event determine all solutions.}
{Define $\alpha = A \cap X, \beta = B \cap \comp{X}, \gamma = C \cap X$ and $\delta = D \cap \comp{X}$. By exercise 5.7a we have $\alpha \cup \beta = \gamma \cup \delta \Leftrightarrow \alpha \cup \beta + \gamma \cup \delta = \emptyset \then \alpha \cup \beta + \gamma \cup \delta = [(\alpha \cup \beta) - (\gamma \cup \delta)] \cup [(\gamma \cup \delta) - (\alpha \cup \beta)] = \emptyset$. By 5.7c we have $[(\alpha \cup \beta) - (\gamma \cup \delta)] = [(\gamma \cup \delta) - (\alpha \cup \beta)] = \emptyset$. Take $x \in B + D$ and assume $x \in A + C$.\\(I) $x \in B \land x \in A \then x \in \alpha \land x \in \beta \then x \in \alpha \cup \beta. x \notin C \land x \notin D \then x \notin \gamma \land x \notin \delta \then x \notin \gamma \cup \delta \then (\alpha \cup \beta) - (\gamma \cup \delta) \neq \emptyset$.\\
(I) $x \in B \land x \in C \then$ (i) $x \in X \then x \notin \alpha \land x \notin \beta \then x \notin \alpha \cup \beta. x \in C \land x \in X \then x \in \gamma \then x \in \gamma \cup \delta \then (\alpha \cup \beta) - (\gamma \cup \delta) \neq \emptyset$.\\(III) similar to (II). (IV) similar to (I). Therefore $B + D \subseteq \comp{A+C}$.}
\end{enumerate}

\hrulefill

\renewcommand{\labelenumi}{1.6.\arabic{enumi}}
\begin{enumerate}

\solution{Show that if $\tuple{x,y,z} = \tuple{u,v,w}$ then $x=u, y=v, z=w$.}
{By definition of ordered triple, $\tuple{x,y,z} = \tuple{\tuple{x,y},z} = \tuple{\tuple{u,v},w}$. Then $z=w$ and $\tuple{x,y} = \tuple{u,v}$. Then from theorem 6.1, we have $x=u$ and $y=v$.}
\solution{Write the members of $\set{1,2} \times \set{2,3,4}$. What are the domain and range of the relation? What is its graph?}
{$\rho = \set{1,2} \times \set{2,3,4} = \set{\tuple{1,2}, \tuple{1,3}, \tuple{1,4}, \tuple{2,2}, \tuple{2,3}, \tuple{2,4}}\\D_{\rho} = \set{1,2} \land R_{\rho} = \set{2,3,4}$}
\item State the domain and range of each of the following relations then draw its graph.
	\begin{enumerate}
	\solution{$\set{\tuple{x,y} \in \mathbb{R}\times\mathbb{R} \mid x^2 + 4y^2 = 1}$}{$D = \set{x\in \mathbb{R} \mid |x| \leq 1} \land R = \set{y \in \mathbb{R} \mid |y| \leq \frac{1}{2}}$}
	\solution{$\set{\tuple{x,y} \in \mathbb{R}\times\mathbb{R} \mid x^2 = y^2}$}{$D = R = \mathbb{R}$.}
	\solution{$\set{\tuple{x,y} \in \mathbb{R}\times\mathbb{R} \mid |x| + 2|y| = 1}$}{...}
	\solution{$\set{\tuple{x,y} \in \mathbb{R}\times\mathbb{R} \mid x^2 + y^2 < 1 \land x > 0}$}{$D = \set{x\in\mathbb{R} \mid 0 < x < 1} \land R = \set{y\in\mathbb{R} \mid |y| < 1}$}
	\solution{$\set{\tuple{x,y} \in \mathbb{R}\times\mathbb{R} \mid y \geq 0 \land y \leq x \land x + y \leq 1}$}
	{$D = \set{x\in\mathbb{R} \mid 0 \leq x \leq 1} \land R = \set{y\in\mathbb{R} \mid 0 \leq y \leq \frac{1}{2}}.$}
	\end{enumerate}
	
\solution{Write the relation in Exercise 6.3(c) as the union of four relations and that in Exercise 6.3(e) as the intersection of three relations.}
{$\rho_{6.3c} = \set{\tuple{x,y}\in\mathbb{R}\times\mathbb{R} \mid y = \frac{1}{2}(1+x)} \cup \set{\tuple{x,y}\in\mathbb{R}\times\mathbb{R} \mid y = \frac{1}{2}(1-x)} \cup \set{\tuple{x,y}\in\mathbb{R}\times\mathbb{R} \mid y = \frac{1}{2}(-1+x)} \cup \set{\tuple{x,y}\in\mathbb{R}\times\mathbb{R} \mid y = \frac{1}{2}(-1-x)}$.\\ $\rho_{6.3e} = \set{\tuple{x,y}\in\mathbb{R}\times\mathbb{R} \mid y \geq 0} \cap \set{\tuple{x,y}\in\mathbb{R}\times\mathbb{R} \mid y \leq x} \cap \set{\tuple{x,y}\in\mathbb{R}\times\mathbb{R} \mid x + y \leq 1}$.}

\solution{The formation of the cartesian product of two sets is a binary operation for sets. Show by examples that this operation is neither commutative nor associative.}
{Define sets $A = \set{1}, B = \set{2}, C = \set{3}$. Then $A \times B = \set{\tuple{1,2}}$ while $B \times A = \set{\tuple{2,1}}$. Therefore the cartesian product is not commutative. Turning to associativity, $(A \times B) \times C = \set{\tuple{1,2}} \times \set{3} = \set{\tuple{\tuple{1,2},3}}$ but $A \times (B \times C) = \set{1} \times \set{\tuple{2,3}} = \set{\tuple{1, \tuple{2,3}}}$. Therefore the cartesian product is not associative.}

\solution{Let $\beta$ be the relation ``is a brother of'', and let $\sigma$ be the relation ``is a sister of''. Describe $\beta \cup \sigma, \beta \cap \sigma,$ and $\beta - \sigma$.}{$\beta \cup \sigma$ corresponds to ``is a brother or a sister of'', namely every pair of siblings. $\beta \cap \sigma = \emptyset$ since someone who is a brother cannot be a sister and vice versa. Because of this we have $\beta - \sigma = \beta$.}

\solution{Let $\beta$ and $\sigma$ have the same meaning as in Exercise 6.6. Let $A$ be the set of students now in the reader's school. What is $\beta[A]$? What is $(\beta \cup \sigma)[A]$?}
{$\beta[A]$ is the set of all the students' brothers. $(\beta \cup \sigma)[A]$ is the set of all the students' siblings.}

\solution{Prove that if $A, B, C$, and $D$ are sets, then $(A \cap B) \times (C \cap D) = (A \times C) \cap (B \times D)$. Deduce that the cartesian multiplication of sets distributes over the operation of intersection, that is, that $(A \cap B) \times C = (A \times C) \cap (B \times C)$ and $A \times (B \cap C) = (A \times B) \cap (A \times C)$ for all $A, B$, and $C$.}
{Let $A,B,C,D$ be arbitrary sets. Then $(A \cap B) \times (C \cap D) = \set{\tuple{x,y} \mid x\in A \cap B \land y \in C \cap D} = \set{\tuple{x,y} \mid x\in A \land x\in B \land y\in C \land y\in D} = \set{\tuple{x,y} \mid x\in A \land y \in C} \cap \set{\tuple{x,y} \mid x\in B \land y \in D} = (A \times C) \cap (B \times D)$. Then as a direct corollary, we have that the cartesian product distributes over set intersection since $(A \cap B) \times C = (A \cap B) \times (C \cap C) = (A \times C) \cap (B \times C)$ and $A \cap (B \times C) = (A \times A) \cap (B \times C) = (A \times B) \cap (A \times C)$.}

\solution{Exhibit four sets $A,B,C$, and $D$ for which $(A \cup B) \times (C \cup D) \neq (A \times C) \cup (B \times D)$.}
{Define sets $A = C = \set{0}$ and $B = D = \set{1}$. Then $(A \cup B) \times (C \cup D) = \set{0,1} \times \set{0,1} = \set{\tuple{0,0},\tuple{0,1},\tuple{1,0},\tuple{1,1}}$ but $(A \times C) \cup (B \times D) = \set{\tuple{0,0}} \cup \set{\tuple{1,1}} = \set{\tuple{0,0}, \tuple{1,1}}$ and so $(A \cup B) \times (C \cup D) \neq (A \times C) \cup (B \times D)$. }

\solution{In spite of the result in the preceding exercise, cartesian multiplication distributes over the operation of union. Prove this.}
{$A \times (B \cup C) = \set{\tuple{a,u} \mid a\in A \land u\in B\cup C} = \set{\tuple{a,u} \mid a\in A \land (u\in B \lor u\in C)} = \set{\tuple{a,b} \mid a\in A \land b\in B} \cup \set{\tuple{a,c} \mid a\in A \land c\in C} = (A \times B) \cup (A \times C)$.\\ $(A \cup B) \times C = \set{\tuple{u,c} \mid u\in A\cup B \land c\in C} = \set{\tuple{u,c} \mid (u\in A \lor u\in B) \land c\in C} = \set{\tuple{a,c} \mid (a\in A \land c\in C} \cup \set{\tuple{b,c} \mid (b\in B \land c\in C} = (A \times C) \cup (B \times C)$.}

\solution{Investigate whether union and intersection distribute over cartesian multiplication.}
{Set union fails to distribute over cartesian multiplication. Take sets $A = B = C = \set{0}$. Then $A \cup (B \times C) = \set{0} \cup \set{\tuple{0,0}} = \set{0, \tuple{0,0}}$ but $(A \cup B) \times (A \cup C) = \set{0} \times \set{0} = \set{\tuple{0,0}}$ and therefore $A \cup (B \times C) \neq (A \cup B) \times (A \cup C)$. A similar derivation will show that $(A \times B) \cup C \neq (A \cup C) \times (B \cup C)$. Set intersection also fails to distribute over cartesian multiplication. We have $A \cap (B \times C) = \set{0} \cap \set{\tuple{0,0}} = \emptyset$ but $(A \cap B) \times (A \cap C) = \set{0} \times \set{0} = \set{\tuple{0,0}}$ and therefore $A \cap (B \times C) \neq (A \cap B) \times (A \cap C)$.}

\solution{Prove that if $A$, $B$, and $C$ are sets such that $A \neq \emptyset$, $B \neq \emptyset$, and $(A \times B) \cup (B \times A) = C \times C$, then $A = B = C$.}
{Let $A, B$, and $C$ be sets such that $A \neq \emptyset$, $B \neq \emptyset$, and $(A \times B) \cup (B \times A) = C \times C$. Then $(A \times B) \cup (B \times A) = \set{\tuple{a,b} \mid a\in A \land b\in B} \cup \set{\tuple{b,a} \mid a\in A \land b\in B} = \set{\tuple{a,b} \mid (a\in A \land b\in B) \lor (a\in B \land b\in A)} = \set{\tuple{c,c} \mid c\in C} \then [x \in C \then (x\in A \land x\in B)] \land [(x\in A \lor x\in B) \then x \in C] \then C \subseteq A \land C \subseteq B \land A \subseteq C \land B \subseteq C \then A = B = C$.}
\end{enumerate}

\hrulefill

\renewcommand{\labelenumi}{1.7.\arabic{enumi}}
\begin{enumerate}
\solution{If $\rho$ is a relation in $\mathbb{R}^+$, then its graph is a set of points in the first quadrant of a coordinate plane. What is the characteristic feature of such a graph if (a) $\rho$ is reflexive, (b) $\rho$ is symmetric, (c) $\rho$ is transitive?}
{(a) $\rho$ contains all points for which $y=x$ (b)  all points either exist on the line $y=x$ or have a complementary point reflected about $y=x$ (c) how to describe??...}

\solution{Using the results of Exercise 7.1, try to formulate a compact characterization of the graph of an equivalence relation on $\mathbb{R}^+$.}
{...}

\solution{The collection of sets $\set{\set{1,3,4},\set{2,7},\set{5,6}}$ is a partition of $\set{1,2,3,4,5,6,7}$. Draw the graph of the accompanying equivalence relation.}
{...}

\solution{Let $\rho$ and $\sigma$ be equivalence relations. Prove that $\rho \cap \sigma$ is an equivalence relation.}
{Let $\rho$ and $\sigma$ be equivalence relations. For all $\tuple{x,y}\in \rho \cap \sigma$ we have $\tuple{x,y}\in\rho$ and $\tuple{x,y}\in\sigma$. Since $\rho$ and $\sigma$ are reflexive, we have $\tuple{x,x}\in\rho$ and $\tuple{x,x}\in\sigma$ and so $\tuple{x,x}\in\rho\cap\sigma$ hence $\rho \cap \sigma$ is reflexive. Since $\rho$ and $\sigma$ are symmetric, we have $\tuple{y,x}\in\rho$ and $\tuple{y,x}\in\sigma$ and therefore $\tuple{y,x}\in\rho\cap\sigma$ hence $\rho \cap \sigma$ is symmetric. For all $\tuple{x,y},\tuple{y,z}\in \rho \cap \sigma$ we have $\tuple{x,y},\tuple{y,z}\in\rho$ and $\tuple{x,y},\tuple{y,z}\in\sigma$ since $\rho$ and $\sigma$ are transitive we have $\tuple{x,z}\in\rho$ and $\tuple{x,z}\in\sigma$ and therefore $\tuple{x,z}\in\rho \cap \sigma$ and so $\rho \cap \sigma$ is transitive. Therefore $\rho \cap \sigma$ is an equivalence relation.}

\solution{Let $\rho$ be an equivalence relation on $X$ and let $Y$ be a set. Show that $\rho \cap (Y \times Y)$ is an equivalence relation on $X \cap Y$.}
{Take any $z\in X\cap Y$. Then $z\in X$ and so $x\in D_{\rho}$. Since $\rho$ is an equivalence relation, we have $\tuple{z,z}\in\rho$. Furthermore, $z\in Y$ and so $\tuple{z,z}\in Y\times Y$. Therefore $\tuple{z,z}\in\rho \cap (Y \times Y)$ and so $\rho \cap (Y \times Y)$ is reflexive. Take any $\tuple{a,b}\in\rho \cap (Y \times Y)$. Then $\tuple{a,b}\in\rho$ and since $\rho$ is symmetric, we also have $\tuple{b,a}\in\rho$. Furthermore, since $\tuple{a,b}\in Y\times Y$ we have $a,b\in Y$ and thus $\tuple{b,a}\in Y\times\rho' Y$. Therefore $\tuple{b,a}\in\rho \cap (Y \times Y)$ and so $\rho \cap (Y \times Y)$ is symmetric. Now take any $\tuple{a,b},\tuple{b,c}\in\rho \cap (Y \times Y)$. Then $\tuple{a,b},\tuple{b,c} \in \rho$ and since $\rho$ is transitive, we have $\tuple{a,c}\in\rho$. Furthermore, since $\tuple{a,b},\tuple{b,c}\in Y\times Y$ we have $a,b,c\in Y$ and so $\tuple{c,a}\in Y\times Y$. Therefore $\tuple{c,a}\in\rho \cap (Y \times Y)$ and so $\rho \cap (Y \times Y)$ is an equivalence relation.}

\item Give an example of these relations.
\begin{enumerate}
	\solution{A relation which is reflexive and symmetric but not transitive.}
	{$\rho = \set{\tuple{a,b} \mid \gcd{a,b} > 1}$}
	\solution{A relation which is reflexive and transitive but not symmetric.}
	{$\rho = \set{\tuple{a,b} \mid a \leq b}$}
	\solution{A relation which is symmetric and transitive but not reflexive in some set.}
	{$\rho = \set{\tuple{a,b} \mid a \text{ is a sibling of } b}$}
\end{enumerate}

\solution{Complete the proof of Theorem 7.1}{See the Notes for a full proof.}

\solution{Each equivalence relation on a set $X$ defines a partition of $X$ according to Theorem 7.1. What equivalence yields the finest partition? the coarsest partition?}{The equivalence which yields the finest partition is that of identity: $\rho = \set{\tuple{x,y} \mid x = y}$. There are $|X|$ equivalence classes in this case. The equivalence which yields the coarsest partition is $\rho = \set{\tuple{x,y} \mid x,y\in X}$ which determines a single equivalence class of size $|X|$.}

\solution{Complete the proof of Theorem 7.2}{See the Notes for a full proof.}

\item Let $\rho$ be a relation which is reflexive and transitive in the set $A$. For $a,b\in A$, define $a\sim b$ iff $a\rho b$ and $b\rho a$.
\begin{enumerate}
	\solution{Show that $\sim$ is an equivalence relation on $A$.}{From the definition of $\sim$ and the reflexivity of $\rho$, we get that $\sim$ is reflexive. Assume $a\sim b$. Then by definition $a\rho b$ and $b\rho a$ and so $b\sim a$. Therefore $\sim$ is symmetric. Assume $a\sim b$ and $b\sim c$. Then $a\rho b$ and $b\rho c$ and by the transitivity of $\rho$, we have $a\rho c$. By symmetry of $\rho$ we also have $c\rho a$. Therefore $a\sim c$ and so $\sim$ is transitive. Hence $\sim$ is an equivalence relation.}
	\solution{For $[a],[b]\in A/\sim$, define $[a]\rho'[b]$ iff $a\rho b$. Show that this definition is independent of $a$ and $b$ in the sense that if $a' \in [a], b' \in [b]$, and $a\rho b$, then $a' \rho b'$.}
	{Take any $a'\in[a]$ and $b'\in[b]$ and assume $a\rho b$. Then $[a]\rho'[b]$. Then since $[a] = [a']$ and $[b] = [b']$ we have $[a']\rho'[b]$ and therefore $a'\rho b'$.}
	\solution{Show that $\rho'$ is reflexive and transitive. Further, show that if $[a]\rho'[b]$ and $[b]\rho'[a]$, then $[a]=[b]$.}
	{Since $\rho$ is transitive, we have $a\rho a$ for all $a\in A$. Therefore $[a]\rho'[a]$ for all $[a]\in A/\sim$ by definition and so $\rho'$ is reflexive. Now assume $[a]\rho'[b]$ and $[b]\rho'[c]$. Then $a\rho b$ and $b\rho c$. By the transitivity of $\rho$ we have $a\rho c$ and thus $[a]\rho'[c]$. Therefore $\rho'$ is transitive. Now assume $[a]\rho'[b]$ and $[b]\rho'[a]$. Then $a\rho b$ and $b\rho a$ and so $a\sim b$. Then we must have that $a$ and $b$ are in the same equivalence class mod $\sim$. Since $a\in[a]$ and $b\in[b]$ we must have $[a]=[b]$.}
\end{enumerate}

\solution{In the set $\mathbb{Z}^+\times\mathbb{Z}^+$ define $\tuple{a,b}\sim\tuple{c,d}$ iff $a+d=b+c$. Show that $\sim$ is an equivalence relation on this set. Indicate the graph of $\mathbb{Z}^+\times\mathbb{Z}^+$, and describe the $\sim$-equivalence classes.}
{For any $x,y\in\mathbb{Z}^+$ we have $x+y=y+x$ and so $\tuple{x,y}\sim\tuple{x,y}$. Therefore $\sim$ is reflexive. By commutativity, $a+d=b+c \iff d+a=c+b \iff c+b=d+a$. Thus whenever $\tuple{a,b}\sim\tuple{c,d}$, we also have $\tuple{c,d}\sim\tuple{a,b}$ and therefore $\sim$ is symmetric. Now assume $\tuple{a,b}\sim\tuple{c,d}$ and $\tuple{c,d}\sim\tuple{e,f}$. Then we have $a+d=b+c$ and $c+f=d+e$. Then $(a+d-b)+f=d+e \then a+f=b+e \iff \tuple{a,b}\sim\tuple{e,f}$ and so $\sim$ is transitive. Therefore $\sim$ is an equivalence relation on $\mathbb{Z}^+\times\mathbb{Z}^+$.}

\end{enumerate}

\hrulefill

\renewcommand{\labelenumi}{1.8.\arabic{enumi}}
\begin{enumerate}

\solution{Give an example of a function on $\mathbb{R}$ into $\mathbb{Z}$.}
{Trivial example: $f:\mathbb{R}\rightarrow\mathbb{Z}$ by $f(x) = 0$ for all $x\in\mathbb{R}$. Non-trivial example: floor function $\floor{\cdot}:\mathbb{R}\rightarrow\mathbb{Z}$ where $\floor{x}$ is the greatest integer less than or equal to $x$. $\floor{x}=n\in\mathbb{Z} \iff n\leq x \land \forall m\in\mathbb{Z}: m\leq x \then m\leq n$. }

\solution{Show that if $A\subseteq X$, then $i_X|A = i_A$}
{Let $A\subseteq X$. Then $i_X|A = i_X(a)$ for $a\in A$. Then clearly $i_X|A$ is the identity mapping on $A$, or $i_X|A = i_A$.}

\solution{If $X$ and $Y$ are sets of $n$ and $m$ elements, respectively, $Y^X$ has how many elements? How many members of $\mathcal{P}(X\times Y)$ are functions?}
{If $X$ and $Y$ are sets of $n$ and $m$ elements, then we have $m$ distinct ways to map each of the $n$ distinct elements of $X$. This amounts to $m^n$ possible maps as the notation $Y^X$ suggests. The members of $\mathcal{P}(X\times Y)$ that are functions are those relations which map all members of $X$ to a member of $Y$ and for which no member of $X$ is mapped to more than one member of $Y$. This is precisely $Y^X$. Thus only $m^n$ out of $2^{m^n}$ (only $|Y^X|=\lg|\mathcal{P}(X\times Y)|$) relations in $\mathcal{P}(X\times Y)$ are functions.}

\item Using only mappings of the form $f:\mathbb{Z}^+\rightarrow\mathbb{Z}^+$, give an example of a function which
\begin{enumerate}
	\solution{is one-to-one but not onto}{$f(x) = 2x$.}
	\solution{is onto but not one-to-one}{$f(x) = \floor{\frac{x+2}{2}}$.}
\end{enumerate}

\solution{Let $A=\set{1,2,\dots,n}$. Prove that if a map $f:A\rightarrow A$ is onto, then it is one-to-one, and if a map $g:A\rightarrow A$ is one-to-one, then it is onto.}{Let $f$ be a map on $A$ onto itself and assume $f(a_1)=f(a_2)$ for some $a_1,a_2\in A$. If $a_1\neq a_2$ then we have two distinct elements in $A$ corresponding to only one element under $f$. This leaves $n-2$ elements to be mapped onto $n-1$ elements. Even if $f|A-(a_1\cup a_2)$ is an onto map, we have one element of $A$ which doesn't get mapped. Then $f$ is not onto, a contradiction. We must have $f(a_1)=f(a_2)\then a_1=a_2$ which means that $f$ is one-to-one.\\Now let $g$ be a one-to-one map on $A$ into itself. Then for any element $b\in A$ we can find a corresponding element $a\in A$ such that $f(a)=b$ because if such an $a$ did not exist, we would have that the $n$ elements of $A$ be mapped into an $A$ subset of size $n-1$. By the pigeonhole principle, we have that two distinct elements must be mapped to the same element, contradicting that $f$ is one-to-one. Therefore $f$ is onto.}

\solution{Let $f:\mathbb{R}^+\rightarrow\mathbb{R}$ where $f(x) = \int_{1}^{x} \frac{dt}{t}$. Show as best you can that $f$ is a one-to-one and onto function.}
{Let $f$ be defined as above as assume $f(x_1)=f(x_2)$ for some $x_1,x_2\in\mathbb{R}^+$ Then the sum of the area under the curve from 1 to $x_1$ equals the area from 1 to $x_2$. But $1/x$ is positively valued for all $x\in\mathbb{R}^+$. Therefore if $x_1<x_2$ we must have that $f(x_1)<f(x_2)$ and similarly for $x_1>x_2$. Thus we must have $x_1=x_2$ and therefore $f$ is one-to-one. Now take any $y\in\mathbb{R}$. Then we can find an $x\in\mathbb{R}^+$, namely $e^y$, such that $f(x) = y$ since we know from calculus that $f(e^y) = \int_{1}^{e^y} \frac{dt}{t} = \ln(e^y) - \ln1 = y\ln e - 0 = y$. Therefore $f$ is onto.}

\solution{Prove that the function $f$ defined in Example 8.7 is a one-to-one correspondence between $\mathcal{P}(X)$ and $2^X$.}
{Let $f$ be defined as in Example 8.7 and suppose $f(A)=f(B)$ for some $A,B\in\mathcal{P}(X)$. Then we have $\chi_A = \chi_B$. Then by definition, $\forall x\in X: (x\in A \iff x\in B)\land (x\notin A \iff x\notin B)$ and so $A=B$. Then $f$ is one-to-one. Now for any $\chi\in 2^X$ we have a coded description of some subset $A\subseteq X$. Then we can easily identify $\chi$ with some member of $\mathcal{P}(X)$ and so $f$ is onto. Therefore $f$ is a one-to-one correspondence between $\mathcal{P}(X)$ and $2^X$.}

\solution{Referring to Example 8.8, prove that if $f$ is a function and $A$ and $B$ are sets, then $f[A\cup B] = f[A] \cup f[B]$.}
{Let $f$ be a function and $A,B$ be sets. (I) Take $y\in f[A\cup B]$. Then there is some $x\in A\cup B$ such that $f(x)=y$. Then $x\in A \then y\in f[A]$ and $x\in B \then y\in f[B]$. Thus $y\in f[A]\cup f[B]$ and so $f[A\cup B] \subseteq f[A] \cup f[B]$. (II) Take $y\in f[A]\cup f[B]$. If $y\in f[A]$ then there exists some $x\in A$ such that $f(x)=y$ and if $y\in f[B]$ then there exists some $x\in B$ such that $f(x)=y$. In both cases $x\in A\cup B$ and so $y\in f[A\cup B]$. Therefore $f[A]\cup f[B] \subseteq f[A\cup B]$. (I) and (II) imply $f[A\cup B] = f[A] \cup f[B]$.}

\solution{Referring to the previous exercise, prove further that $f[A\cap B] \subseteq f[A]\cap f[B]$, and show that proper inclusion can occur.}
{Let $f$ be a function and $A,B$ be sets. (I) Take $y\in f[A\cap B]$. Then there exists an $x\in A\cap B$ such that $f(x)=y$. Then $x\in A$ and $x\in B$ and so $y\in f[A]$ and $y\in f[B]$ and thus $y\in f[A]\cap f[B]$. Therefore $f[A\cap B]\subseteq f[A]\cap f[B]$. (II) Take $y\in f[A]\cap f[B]$. Since $y\in f[A]$ there exists an $x_1\in A$ such that $f(x_1)=y$ and since $y\in f[B]$ there exists an $x_2\in B$ such that $f(x_2)=y$. In the case that $x_1=x_2$ then we have $x_1\in A\cap B$ and so $y\in f[A\cap B]$. Therefore $f[A]\cap f[B]\subseteq f[A\cap B]$. In this case (I) and (II) imply $f[A\cap B] = f[A]\cap f[B]$.}

\solution{Prove that a function $f$ is one-to-one iff for all sets $A$ and $B$, $f[A\cap B] = f[A] \cap f[B]$.}
{($\Rightarrow$) Let $f$ be a one-to-one function. Let $A$ and $B$ be sets. We must show $f[A]\cap f[B]\subseteq f[A\cap B]$. Take $y\in f[A]\cap f[B]$. Since $y\in f[A]$ there exists an $x_1\in A$ such that $f(x_1)=y$ and since $y\in f[B]$ there exists an $x_2\in B$ such that $f(x_2)=y$. But since $f$ is one-to-one we must have that $x_1=x_2$. By the previous exercise, we get that $f[A]\cap f[B]\subseteq f[A\cap B]$.\\($\Leftarrow$) Let $f$ be a function such that $f[A]\cap f[B]\subseteq f[A\cap B]$ for all sets $A$ and $B$. Assume $f(a)=f(b)$ for some $a\in A$ and $b\in B$. Then from the previous exercise we have that $a=b$ and so $f$ is one-to-one.}

\solution{Prove that a function $f:X\rightarrow Y$ is onto $Y$ iff $f[X-A]\supseteq Y-f[A]$ for all sets $A$.}
{($\Rightarrow$) Let $f$ be a function on $X$ onto $Y$. Take $y\in Y-f[A]$. Then $y\in Y$ and $y\notin f[A]$. Then there is an $x\notin A$ such that $f(x)=y$. So $x\in X-A$ and therefore $f[X-A]\supseteq Y-f[A]$.\\($\Leftarrow$) Let $f$ be a function on $X$ into $Y$ such that $f[X-A]\supseteq Y-f[A]$ for all sets $A$. Then taking $A=\emptyset$ we have $f[X]\supseteq Y$. Then for any $y\in Y$ we can find an $x\in X$ such that $f(x)=y$ and so $f$ is onto.}

\solution{Prove that a function $f:X\rightarrow Y$ is one-to-one and onto iff $f[X-A]=Y-f[A]$ for all sets $A$.}
{($\Rightarrow$) Let $f:X\rightarrow Y$ be a one-to-one and onto function. Take $y\in f[X-A]$ for some set $A$. Then there exists an $x\in X-A$ such that $f(x)=y$ and since $y$ is one-to-one this is the only such $x\in X$. Then $x\notin A \then y\notin f[A] \then y\in Y-f[A]$. Therefore $f[X-A]\subseteq Y-f[A]$. Because $f$ is onto we have from the previous exercise that $f[X-A]\supseteq Y-f[A]$. Together, we have $f[X-A]=Y-f[A]$ for all sets $A$.\\($\Leftarrow$) Let $f:X\rightarrow Y$ be a function such that $f[X-A]=Y-f[A]$ for all sets $A$. Since $f[X-A]\supseteq Y-f[A]$ for all sets $A$, we have from the previous exercise that $f$ is onto. It remains to show that $f$ is one-to-one. Assume $f(x_1)=f(x_2)$ for $x_1,x_2\in X$. Then we have $f[X-\set{x_1}]=Y-f(x_1)$ and $f[X-\set{x_2}]=Y-f(x_2)$ and so $f[X-\set{x_1}]=f[X-\set{x_2}]$. But $x_2\in X-\set{x_1}$ and so $f(x_2)\in f[X-\set{x_1}$ but $x_2\notin X-\set{x_2}$ and so $f(x_2)\notin f[X-\set{x_2}]$, a contradiction. Then we must have $x_1=x_2$ and so $f$ is one-to-one.}

\end{enumerate}

\hrulefill

\renewcommand{\labelenumi}{1.9.\arabic{enumi}}
\begin{enumerate}

\solution{Let $f:\REAL\goesto\REAL$ where $f(x)=\left(1+(1-x)^{1/3}\right)^{1/5}$. Express $f$ as the composite of four functions, none of which is the identity function.}{Define $f_1(x) := 1 - x$, $f_2(x) := x^{1/3}$, $f_3(x) := 1 + x$, and $f_4(x) := x^{1/5}$. Then $$f=f_4\circ f_3\circ f_2\circ f_1.$$}

\solution{If $f:X\goesto Y$ and $A\subseteq X$, show that $f\mid A=f\circ i_A$.}{Let $y\in f\circ i_A$. Then $y=f(i_A(x))$ for some $x\in X$. Since the domain of $i_A$ is $A$ we must have $x\in A$. Then $y=f(i_A(x))=f(x)$ for some $x\in A$ and so $y\in f\mid A$. Now let $y\in f\mid A$. Then $y=f(x)$ for some $x\in A$. Since $x=i_A(x)$ for all $x\in A$, $y=f(i_A(x))=(f\circ i_A)(x)$ and therefore $y\in f\circ i_A$.}

\solution{Complete the proof of the assertions made in Example 9.2.}{If $\rho$ is an equivalence relation and $D_{\rho}=X$, then $$j:X\goesto X/\rho$$ with $j(x)=[x]$ is onto the quotient set $X/\rho$ ($\Surj{j}$); $j$ is called the \textbf{canonical} or \textbf{natural mapping} on $X$ onto $X/\rho$. If $f:X\goesto Y$ then the relation $\rho$ defined by $$x_1\rho x_2 \iff f(x_1)=f(x_2)$$ is an equivalence relation on $X$.\\

\small{\underline{Proof}:\\
$\begin{array}{ccc}\forall x: f(x)=f(x) \then x\rho x.\\ \forall x,y: [f(x)=f(y) \iff f(y)=f(x)] \then [x\rho y \iff y\rho x].\\ \forall x,y,z: [f(x)=f(y)\land f(y)=f(z)\iff f(x)=f(z)] \then [x\rho y \land y\rho z \iff x\rho z].\end{array}$}\\

Let $j$ be the canonical map on $X$ onto $X/\rho$. We contend that a function $g$ on $X/\rho$ into $f[X]$, the range of $f$, is defined by setting $g([x])=f(x)$. To see that $g$ is a function, first note that $R_g\subseteq R_f\then g(X)\subseteq Y \iff R_g \subseteq Y \iff C_g = Y$, where $C_g$ is the codomain of $g$. To see that $g$ is well-defined, consider $x,y\in X$. Then $x\rho y\iff[x]=[y]\iff f(x)=f(y) \iff g([x])=g([y])$. Finally, let $i$ be the injection of $f[X]$ into $Y$. Then we have defined function $j, g, i$, where
\begin{align*}j:&X\goesto X/\rho :: j(x) = [x]\\ g:&X/\rho\goesto f[X] :: g([x])=f(x)\\ i:&f[X]\goesto Y :: i(y) = y\end{align*} for a function $f:X\goesto Y$. Then $\Surj{j}$ and $\Inj{i}$. 

\small{\underline{Proof}:\\
$\forall [x]\in X/\rho: \exists x \in X: j(x) = [x]$ (Take element $x$ which is canonical representative of equivalence class $[x]$.) $\then \Surj{j}$.
$\forall y,z\in Y: y=z \then i(y) = y = z = i(z) \then \Inj{i}$.}

$\Bij{g}$. \underline{Proof}: $\forall [x],[y]\in X/\rho: [x]\neq[y] \iff \lnot x\rho y \iff f(x)\neq f(y)\iff g([x])\neq g([y])\then \Inj{g}.$ $\forall y\in f[X] \exists x\in X/\rho: f(x)=y \then f([x])=y \then \exists [x]\in X/\rho: g([x])=f(x)=y \then \Surj{g}$. $\Inj{g}\land\Surj{g}\then\Bij{g}$.
By associativity, $i\circ g \circ j = i\circ(g\circ j)$. By definition, $$g\circ j:X\goesto f[X]:: (g\circ j)(x) = g(j(x))=g([x])=f(x)$$ and $$i\circ(g\circ j):X\goesto Y :: (i\circ (g\circ j))(x)=i((g\circ j)(x))=i(f(x))=f(x).$$ Therefore $f = i\circ g \circ j$.}


\solution{Complete the proof of (I) and supply a proof of (II) in Example 9.3.}{(I) Let $f:X\goesto Y$. Then $\Inj{f}$ iff for all functions $g$ and $h$ such that $g:Z\goesto X$ and $h:Z\goesto X$, $f\circ g=f\circ h$ implies that $g=h$. \underline{Proof}: Suppose that $\Inj{f}$ and that $g$ and $h$ are mappings on $Z$ into $X$ for which $f\circ g=f\circ h$. Then $f(g(z)) = f(h(z))$ for all $z\in Z$. With $\Inj{f}$ it follows that $g(z)=h(z)$ for all $z\in Z$. Hence $g=h$. Now suppose for all functions $g$ and $h$ such that $g:Z\goesto X$ and $h:Z\goesto X$, $f\circ g=f\circ h$ implies that $g=h$. Suppose $g(z)=x_1$ and $h(z)=x_2$ for some $z\in Z$ and $x_1,x_2\in X$. Then $[(f\circ g)(z)=(f\circ h)(z) \then g(z)=h(z)] \iff [f(g(z))=f(h(z))\then g(z)=h(z)]\iff [f(x_1)=f(x_2)\then x_1=x_2]\iff \Inj{f}$.

(II) Let $f:X\goesto Y$. Then $\Surj{f}$ iff for all functions $g$ and $h$ such that $g:Y\goesto Z$ and $h:Y\goesto Z$, $g\circ f=h\circ f$ implies $g=h$. \underline{Proof}: Suppose $\Surj{f}$ and suppose $g$ and $h$ are mappings from $Y$ into $Z$ such that $g\circ f=h \circ f$. Then $(g\circ f)(x)=(h\circ f)(x)$ for all $x\in X$. And so $g(f(x))=h(f(x))$ for all $f(x)\in Y$. That is, for all $y\in f[X]$, $g(y)=h(y)$. Since $\Surj{f}$ we have that $f[X]=Y$ and so $g(y)=h(y)$ for all $y\in Y$ and so $g=h$. Now suppose for all functions $g$ and $h$ such that $g:Y\goesto Z$ and $h:Y\goesto Z$, $g\circ f=h\circ f$ implies $g=h$. Suppose there is a $y\in Y$ such that $f(x)\neq y$ for all $x\in X$. In other words, $f[X]\neq Y$ or $\lnot\Surj{f}$. Then suppose $g(y)=z_1$, $h(y)=z_2$, and $g(y')=h(y')$ for all other $y'\in Y$. Then $g\circ f= h\circ f$ since $g$ and $h$ only differ at the evaluation of $y$. But $g\neq h$ so our assumption is violated. Therefore $\Surj{f}$.}

\solution{Prove that $f:A\goesto B$ is a one-to-one correspondence between $A$ and $B$ iff there exists a map $g:B\goesto A$ such that $g\circ f = i_A$ and $f\circ g = i_B$.}{Let $f:A\goesto B$ be a one-to-one correspondence between $A$ and $B$. Then there exists an inverse $f^{-1}:B\goesto A$ such that $f^{-1}\circ f=i_A$ and $f \circ f^{-1}=i_B$. Letting $g=f^{-1}$ we satisfy the right implication. Suppose there exists a map $g:B\goesto A$ such that $g\circ f = i_A$ and $f\circ g = i_B$.  Then $g(f(a))=a$ for all $a\in A$ and so $g=f^{-1}$ and $f(g(b))=b$ for all $b\in B$ and so $f=g^{-1}$. Therefore $f$ and $g$ are both invertible and so are one-to-one correspondences.}

\solution{If $f:A\goesto B$ and $g:B\goesto C$ are both one-to-one and onto, show that $g\circ f:A\goesto C$ is one-to-one and onto and that $(g\circ f)^{-1}=f^{-1}\circ g^{-1}$.}{Suppose $\Bij{f}\land\Bij{g}$. For all $a_1\neq a_2\in A$ we have $f(a_1)\neq f(a_2)$ since $\Inj{f}$. Then since $\Inj{g}$ we have $(g \circ f)(a_1)=g(f(a_1))\neq g(f(a_2))=(g\circ f)(a_2)$ and so $a_1\neq a_2\then (g\circ f)(a_1)\neq(g\circ f)(a_2)$. Therefore $\Inj{g\circ f}$. For any $c\in C$, there is a $b\in B$ with $g(b)=c$ since $\Surj{g}$. For any $b\in B$, there is an $a\in A$ with $f(a)=b$ since $\Surj{f}$. Then $c=g(b)=g(f(a))=(g\circ f)(a)$ for some $a\in A$. Therefore $\Surj{g\circ f}$. Since $\Inj{g\circ f}\land\Surj{g\circ f}$ we have $\Bij{g\circ f}$ and so we can define an inverse $(g\circ f)^{-1}:C\goesto A$ such that $(g\circ f)^{-1}\circ(g\circ f)=i_A$. Then for all $a\in A$, $[(g\circ f)^{-1}\circ(g\circ f)](a)= (g\circ f)^{-1} \circ g(f(a)) = i_A$. But $(f^{-1}\circ g^{-1})\circ(g(f(a)))=f^{-1}(g^{-1}(g(f(a))))=f^{-1}(f(a))=a=i_A$ and so $(g\circ f)^{-1} =f^{-1}\circ g^{-1}$.}

\solution{For a function $f:A\goesto A$, $f^n$ is the standard abbreviation for $f\circ f\circ \dots \circ f$ with $n$ occurrences of $f$. Suppose that $f^n=i_A$. Show that $f$ is one-to-one and onto.}{Let $f:A\goesto A$ and suppose $f^n=i_A$. Then for all $a_1\neq a_2\in A$ and suppose $f(a_1)=f(a_2)=a_3$. Then $f^n(f(a_1))=f^{n+1}(a_1)=a_3$ and $f^n(f(a_2))=f^{n+1}(a_2)=a_3$. But $f^n(a_1)=a_1$ and $f^n(a_2)=a_2$ and so $f(a_1)=a_1$ and $f(a_2)=a_2$. Since it's assumed that $a_1\neq a_2$ this is a contradiction. Then $a_1\neq a_2\in A \then f(a_1)\neq f(a_2)$ and so $\Inj{f}$. For any $a\in A$ we have $f^{n-1}(a)\in A$ and $f(f^{n-1}(a)) = f^n(a)=a$. Then $\Surj{f}$.}

\solution{Justify the following restatement of Theorem 7.1. Let $X$ be a set. Then there exists a one-to-one correspondence between the equivalence relations on $X$ and the partitions of $X$.}{Define a function $f$ from the set of equivalence relations on $X$ to the set of partitions of $X$ such that the image of a given equivalence relation under $f$ is the set of its equivalence classes. Theorem 7.1 states that this image is a partition of $X$. Showing that an equivalence relation determines a partition of $X$ and vice versa is equivalent to showing that no two distinct equivalence classes map to the same partition ($\Inj{f}$) and that every partition has a corresponding equivalence relation ($\Surj{f}$). Therefore the statement of Theorem 7.1 is equivalent to the saying $\Bij{f}$.}

\solution{Prove that if the inverse of the function $f$ in $\REAL$ exists, then the graph of $f^{-1}$ may be obtained from that of $f$ by reflection in the line $y=x$.}{If the inverse of $f$ in $\REAL$ exists, then for all $x\in\REAL$ we have $f(x)=y$ for some $y\in\REAL$ and $f^{-1}(y)=x$. Therefore the roles of independent and dependent variable are swapped out for $f^{-1}$. Having a graph with $x$ plotted on one axis and $y$ an a second axis perpendicular to the first, we must flip the graph $180^{\circ}$ around the line $y=x$ to achieve this variable swap graphically.}

\item Show that each of the following functions has an inverse. Determine the domain of each inverse and its value at each member of its domain. Further, sketch the graph of each inverse.
\begin{enumerate}
\solution{$f:\REAL\goesto\REAL$ where $f(x)=2x-1$.}{$\forall x,y\in\REAL$, $x\neq y\then f(x)=2x-1\neq 2y-1=f(y)\then \Inj{f}$ and $\forall y\in\REAL$, $f(\frac{1}{2}(y+1))=2(\frac{1}{2}(y+1))-1=y+1-1=y\then\Surj{f}$. Then $D_{f^{-1}}=\REAL$ and $f^{-1}(x)=\frac{1}{2}(x+1)$.}
\solution{$f:\REAL\goesto\REAL$ where $f(x)=x^3$.}{$\forall x,y\in\REAL$, $x\neq y\then f(x)=x^3\neq y^3=f(y)\then\Inj{f}$ and $\forall y\in\REAL$, $f(\sqrt[3]{y})=(\sqrt[3]{y})^3=y\then\Surj{y}$. Then $D_{f^{-1}}=\REAL$ and $f^{-1}(x)=\sqrt[3]{x}$.}
\solution{$f=\set{\tuple{x,(1-x^2)^{1/2}} \mid 0\leq x\leq 1}$}{$\forall x,y\in[0,1]$, if $x\neq y$ then $f(x)=(1-x^2)^{1/2}\neq (1-y^2)^{1/2}=f(y)$ since $g(x)=x^2$ and $h(x)=x^{1/2}$ are both monotonically increasing in $[0,1]$. So $\Inj{f}$. For all $y\in[0,1]$ we have $f((1-y^2)^{1/2})=(1-((1-y^2)^{1/2})^2)^{1/2}=(1-(1-y^2))^{1/2}=(y^2)^{1/2}=y$. Then $D_{f^{-1}}=[0,1]$ and $f^{-1}(x)=(1-x^2)^{1/2}=f(x)$.}
\solution{$f=\set{\tuple{x,\frac{x}{x-1}\mid -2\leq x < 1}}$}{$\forall x,y\in[-2,1)$, if $x\neq y$ then $f(x)=\frac{x}{x-1}\neq\frac{y}{y-1}=f(y)\then\Inj{f}$. $\forall y\in[-2,1)$ we have $f(\frac{y}{y-1})=\frac{y/(y-1)}{y/(y-1) - 1}=\frac{y/y-1}{1/y-1}=y\then\Surj{f}$. Then $D_{f^{-1}}=[-2,1)$ and $f^{-1}(x)=\frac{x}{x-1}=f(x)$.}
\end{enumerate}















\end{enumerate}




\end{document}
